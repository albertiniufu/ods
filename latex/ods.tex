\documentclass[10pt]{book}
\usepackage{makeidx}
\usepackage{framed}
\setlength{\FrameSep}{0mm}
 
\makeindex
%\usepackage{showidx}
% mark overful hboxes
%\overfullrule=5mm
\usepackage{kpfonts}

\newcommand{\eq}{=}

%\usepackage[nott]{kpfonts}
%\SetMathAlphabet{\mathtt}{normal}{OT1}{\ttdefault}{m}{n}
%\SetMathAlphabet{\mathtt}{bold}{OT1}{\ttdefault}{m}{n}

%\usepackage[math]{iwona}
%\SetMathAlphabet{\mathtt}{iwona}{OT1}{\ttdefault}{m}{n}
%\usepackage[T1]{fontenc}

\usepackage{amsopn}
\usepackage{amsmath}
\usepackage{amsthm}
\usepackage{url}
\usepackage{graphicx}
\usepackage{datetime}
\usepackage{amsfonts}
\usepackage{graphicx}
\usepackage{threeparttable}
\usepackage{wasysym}
\usepackage{emptypage}
\usepackage{titling}
%\usepackage{calc}


%\usepackage[mathlines]{lineno}
%\linenumbers
%\DeclareGraphicsExtensions{.pdf,.eps}

% Leave this here - it gets substituted with language specific stuff
%HEADCOMMAND

\allowdisplaybreaks[1]  % for ams math align environments

\hyphenation{Array-Stack}
\hyphenation{Fast-Array-Stack}
\hyphenation{Array-Queue}
\hyphenation{Array-Deque}
\hyphenation{Dual-Array-Deque}
\hyphenation{Root-ish-Array-Stack}
\hyphenation{Skip-list-Set}
\hyphenation{Skip-list-List}
\hyphenation{Hash-Table}
\hyphenation{Chained-Hash-Table}
\hyphenation{Linear-Hash-Table}
\hyphenation{Red-Black-Tree}
\hyphenation{Binary-Tree}
\hyphenation{Binary-Search-Tree}
\hyphenation{Scape-goat-Tree}
\hyphenation{Count-down-Tree}
\hyphenation{Dy-na-mite-Tree}
\hyphenation{Binary-Heap}
\hyphenation{Meld-able-Heap}
\hyphenation{Java-Script}

\usepackage{everysel}
\EverySelectfont{%
%\fontdimen2\font=0.4em% interword space
%\fontdimen3\font=0.2em% interword stretch
%\fontdimen4\font=0.1em% interword shrink
%\fontdimen7\font=0.1em% extra space
\hyphenchar\font=`\-% to allow hyphenation
}

\usepackage[sf,small,raggedright]{titlesec} % formatting titles
\titlespacing*{\section}{0pt}{24pt}{14pt}
\titlespacing*{\subsection}{0pt}{14pt}{14pt}
\usepackage{relsize,fancyvrb}  % formatting pseudocode
\usepackage{ods} % Personalization and commands

% These command are expanded by scripts, otherwise they should be ignored
\newcommand{\javaimport}[1]{}
\newcommand{\cppimport}[1]{}
\newcommand{\pcodeimport}[1]{}

\htmlonly{
  \newcommand{\ScaleIfNeeded}{\textwidth}
  \newcommand{\HalfScaleIfNeeded}{\textwidth}
  \newcommand{\HeightScaleIfNeeded}{\textheight}
  \newcommand{\QuarterHeightScaleIfNeeded}{.25\textheight}
  \newcommand{\FifthHeightScaleIfNeeded}{.2\textheight}
  \newcommand{\fancyhead}[2][zzz]{}
  \newcommand{\fancyfoot}[2][zzz]{}
}

% Referencing commands 
\newcommand{\chaplabel}[1]{\label{chap:#1}}
\newcommand{\Chapref}[1]{Capítulo~\ref{chap:#1}}
\newcommand{\chapref}[1]{Capítulo~\ref{chap:#1}}
\newcommand{\seclabel}[1]{\label{sec:#1}}
\newcommand{\Secref}[1]{Seção~\ref{sec:#1}}
\newcommand{\secref}[1]{Seção~\ref{sec:#1}}
\newcommand{\sref}[1]{\textsection~\ref{sec:#1}}

\newcommand{\alglabel}[1]{\label{alg:#1}}
\newcommand{\Algref}[1]{Algoritmo~\ref{alg:#1}}
\newcommand{\algref}[1]{Algoritmo~\ref{alg:#1}}

\newcommand{\applabel}[1]{\label{app:#1}}
\newcommand{\Appref}[1]{Apêndice~\ref{app:#1}}
\newcommand{\appref}[1]{Apêndice~\ref{app:#1}}

\newcommand{\tablabel}[1]{\label{tab:#1}}
\newcommand{\Tabref}[1]{Tabela~\ref{tab:#1}}
\newcommand{\tabref}[1]{Tabela~\ref{tab:#1}}

\newcommand{\figlabel}[1]{\label{fig:#1}}
\newcommand{\Figref}[1]{Figura~\ref{fig:#1}}
\newcommand{\figref}[1]{Figura~\ref{fig:#1}}

\newcommand{\eqlabel}[1]{\label{eq:#1}}
\newcommand{\myeqref}[1]{(\ref{eq:#1})}
\newcommand{\Eqref}[1]{Equação~(\ref{eq:#1})}

% Theorem-like environments
\theoremstyle{plain}
\newtheorem{thm}{Theorem}[chapter]
\newcommand{\thmlabel}[1]{\label{thm:#1}}
\newcommand{\thmref}[1]{Teorema~\ref{thm:#1}}

\newtheorem{lem}{Lemma}[chapter]
\newcommand{\lemlabel}[1]{\label{lem:#1}}
\newcommand{\lemref}[1]{Lema~\ref{lem:#1}}

\newtheorem{cor}{Corollary}[chapter]
\newcommand{\corlabel}[1]{\label{cor:#1}}
\newcommand{\corref}[1]{Corolário~\ref{cor:#1}}

\theoremstyle{definition}

\newtheorem{exc}{Exercise}[chapter]
\newcommand{\exclabel}[1]{\label{exc:#1}}
\newcommand{\excref}[1]{Exercício~\ref{exc:#1}}


\newtheorem{prp}{Property}[chapter]
\newcommand{\prplabel}[1]{\label{prp:#1}}
\newcommand{\prpref}[1]{Propriedade~\ref{prp:#1}}

% Miscellaneous commands
\newcommand{\etal}{\emph{et al.}}
\newcommand{\voronoi}{Vorono\u\i}
\newcommand{\ceil}[1]{{\lceil #1 \rceil}}
\newcommand{\Ceil}[1]{{\left\lceil #1 \right\rceil}}
\newcommand{\floor}[1]{{\lfloor #1 \rfloor}}
\newcommand{\Floor}[1]{{\left\lfloor #1 \right\rfloor}}
\newcommand{\R}{\mathbb{R}}
\newcommand{\N}{\mathbb{N}}
\newcommand{\Z}{\mathbb{Z}}
\newcommand{\Sp}{\mathbb{S}}
\newcommand{\E}{\mathrm{E}}
\DeclareMathOperator{\ddiv}{div}

\usepackage{ods-colors}

\usepackage{tikz,gnuplot-lua-tikz}

% The following is a work-around for bad distribution of gnuplot-tex
% https://bugs.debian.org/cgi-bin/bugreport.cgi?bug=835028
\def\gpsetdashtype#1{}  

\usepackage[bookmarks]{hyperref}
\hypersetup{colorlinks=true, linkcolor=linkblue,  anchorcolor=linkblue,%
	citecolor=linkblue, filecolor=linkblue, menucolor=linkblue,%
	urlcolor=linkblue,%
    pdfauthor={Pat Morin. Tradução e adaptação: Marcelo Keese Albertini},%
    pdftitle={Open Data Structures. Estruturas de Dados Abertas},%
    pdfsubject={Computer Science, Data Structures},%
    pdfkeywords={Data structures, algorithms}} 

\DeclareMathOperator{\bdiv}{div}

% Title page content
\title{Open Data Structures (in \lang). Estruturas de Dados Abertas (em \lang)}
\author{Autor: Pat Morin.\\Tradução e adaptações por: Marcelo Keese Albertini}
\date{%
Edition 0.1G\cpponly{$\beta$}\pcodeonly{$\beta$}
\htmlonly{\\ \includegraphics[scale=0.90909,scale=0.5]{images/cc-by}}}
%Version 0.0 pre $\alpha$: \today}

\pagenumbering{roman}

% Draft mode only - mark overfull hboxes
% \overfullrule=5pt

\begin{document}

%%\AddToShipoutPicture*{\BackgroundPic}
\htmlonly{\newcommand{\thetitlepage}{
  \begin{center}\includegraphics[scale=0.90909]{images/tree3-thick}\end{center}
  \maketitle
}}
\thetitlepage

\cleardoublepage
%
%% blank page behind title page
%\ \thispagestyle{empty}\newpage
%
%\setcounter{page}{1}
%\chapter*{Agradecimentos}
\addcontentsline{toc}{chapter}{Agradecimentos}

Agradeço à todos que ajudaram e motivaram a fazer esta tradução. Devido à natureza pessoal de agradecimentos, vou manter os agradecimentos originais deste livro a seguir.

I am grateful to Nima~Hoda, who spent a summer tirelessly proofreading
many of the chapters in this book; to the students in the Fall 2011
offering of COMP2402/2002, who put up with the first draft of this book
and spotted many typographic, grammatical, and factual errors; and to
Morgan~Tunzelmann at Athabasca University Press, for patiently editing
several near-final drafts.

%\ \thispagestyle{empty}\newpage
%\cpponly{\include{cpp-preface}
%\ \thispagestyle{empty}\newpage
%}

% Use 14pt between lines
\setlength{\baselineskip}{14pt}


%\begin{titlepage}
%\maketitle
%\end{titlepage}

%\pagestyle{empty}
%half title page
%\newpage
%
%series page
%\newpage
%
%title page
%\newpage

\addtocontents{toc}{\protect\thispagestyle{empty}} % get rid of page number
\tableofcontents
\cleardoublepage

\fancyhead[RO,LE]{} % disable section numbers, for now
\pagestyle{fancy}
\chapter*{Agradecimentos}
\addcontentsline{toc}{chapter}{Agradecimentos}

Agradeço à todos que ajudaram e motivaram a fazer esta tradução. Devido à natureza pessoal de agradecimentos, vou manter os agradecimentos originais deste livro a seguir.

I am grateful to Nima~Hoda, who spent a summer tirelessly proofreading
many of the chapters in this book; to the students in the Fall 2011
offering of COMP2402/2002, who put up with the first draft of this book
and spotted many typographic, grammatical, and factual errors; and to
Morgan~Tunzelmann at Athabasca University Press, for patiently editing
several near-final drafts.

\thispagestyle{empty}
\cleardoublepage

\fancyhead[CE]{\small Why This Book?} % chapter title, left center
\chapter*{Por que este livro?}
\addcontentsline{toc}{chapter}{Por que este livro?}

Existem muitos livros introdutórios à disciplina de estruturas de dados.
Alguns deles são muito bons. A maior parte deles são pagos e a
grande maioridade de estudantes de Ciência da Computação e Sistemas de
Informação irá gastar algum dinheiro em um livro de estruturas de dados.

Vários livros gratuitos sobre estruturas de dados estão disponíveis online.
Alguns muito bons, mas a maior parte está ficando desatualizada. Grande
parte deles torna-se gratuita quando seus autores e/ou editores decidem
parar de atualizá-los. Atualizar esses livros frequentemente não é possível,
por duas razões: (1)~Os direitos autorais pertencem ao autor e/ou editor,
os quais podem não autorizar tais atualizações. (2)~O \emph{código-fonte} desses
livros muitas vezes não está disponível. Isto é, os arquivos Word, WordPerfect,
FrameMaker ou \LaTeX\ para o livro não são acessíveis e, ainda, a versão do
software que processa esses arquivos pode não estar mais disponível.

A meta deste projeto é libertar estudantes de graduação de Ciência da Computação
de ter que pagar por um livro introdutório à disciplina de estruturas de dados.

Eu
\footnote{The translator to Portuguese language is deeply grateful to the original author of this book Pat Morin for his decision which allows the availability of a good quality and free book in the
native language of my Brazilian students.} decidi implementar essa meta ao tratar esse livro como um projeto de Software Aberto%
\index{Open Source}%
\index{Software Aberta}%
.

Os arquivos-fonte originais em \LaTeX, \lang\ e scripts para montar este livro estão disponíveis para download a partir do website do autor\footnote{\url{http://opendatastructures.org}}
e também, de modo mais importante, em um site confiável de gerenciamento de códigos-fonte.\footnote{\url{https://github.com/patmorin/ods}}\footnote{Tradução em português em \url{https://github.com/albertiniufu/ods}}

O código-fonte disponível é publicado sob uma licença
Creative Commons Attribution,
o que quer dizer que qualquer um é livre para \emph{compartilhar}:
\index{compartilhar}
copiar, distribuir e
transmitir essa obra; e \emph{modificar}:
\index{modificar}
adaptar a obra, incluindo o direito
de fazer uso comercial da obra.
A única condição a esse direitos é \emph{atribuição}:
você deve reconhecer que a obra derivada contém código e/ou texto de \url{opendatastructures.org}.

Qualquer pessoa pode contribuir com correções usando o sistema de gerenciamento
de códigos-fonte \texttt{git}%
\index{git@\texttt{git}}%
. Qualquer pessoa também pode fazer fork (criar versão alternativa) dos arquivos-fonte deste livro e desenvolver uma versão separada (por exemplo, em outra linguagem de programação).
Minha esperança é que, ao assim fazer, este livro continuará a
ser didático mesmo depois que o meu interesse no projeto, ou meu batimento cardíaco, desapareça (o que quer que aconteça primeiro).


\cleardoublepage

\cpponly{
  \include{cpp-preface}
  \cleardoublepage
}

\fancyhead[CE]{\small\nouppercase{\leftmark}} % chapter title, left center
\fancyhead[CO]{\small\rightmarktitle} % section title, right center
\fancyhead[RO,LE]{\small\rightmarksection}

%% Include all the chapters one at a time
\include{intro-lang}
\include{arrays-lang}
\include{linkedlists-lang}
\include{skiplists-lang}
\include{hashing-lang}
\include{binarytrees-lang}
\include{rbs-lang}
\include{scapegoat-lang}
\include{redblack-lang}
\include{heaps-lang}
\include{sorting-lang}
\include{graphs-lang}
\include{integers-lang}
\include{btree-lang}

%% Turn off section numbers for remainder of document
\fancyhead[RO]{} % section number on the outside
\fancyhead[LE]{} % section number on the outside
\renewcommand{\chaptermark}[1]{\markboth{#1}{}} 
\renewcommand{\sectionmark}[1]{\markright{#1}} 
\fancyhead[CO]{\small\nouppercase\rightmark}

\cleardoublepage
\addcontentsline{toc}{chapter}{Bibliography}
\bibliographystyle{abbrvurl}
\bibliography{ods,odsproc}

\cleardoublepage
\addcontentsline{toc}{chapter}{Index}
\printindex

\end{document}

