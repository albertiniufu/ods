\chapter{Tabelas Hash }
\chaplabel{hashtables}
\chaplabel{hashing}

Tabelas hash são um eficiente método de guardar um pequeno número #n#
de inteiros provenientes de um grande intervalo
$U=\{0,\ldots,2^{#w#}-1\}$.
O termo tabela hash, ou \emph{hash table}, 
\index{hash table}%
\index{tabela hash}%
inclui um grande número de tipos de estruturas de dados. A primeira parte
deste capítulo foca em duas das implementações mais comuns de tabelas hash:
hashing com encadeamento e sondagem linear.

Muito frequentemente, tabelas hash guardam tipos de dados que não são inteiros.
Nesse caso, um inteiro chamado de código \emph{hash}
\index{código hash}%
é associado com cada item de dados e é usado na tabela hash.
A segunda parte deste capítulo discute como tais códigos hash
são gerados.

Alguns desses métodos que usados neste capítulo requerem obter
inteiros aleatórios em algum intervalo. Nos trechos de código,
alguns desses inteiros ``aleatórios'' são constantes fixas no código.
Essas constantes foram obtidas usando bits aleatórios gerados a partir
de ruído atmosférico.

\section{#ChainedHashTable#: Hashing com Encadeamento}
\seclabel{hashtable}

\index{ChainedHashTable@#ChainedHashTable#}%
\index{encadeamento}%
\index{hashing com encadeamento}%
Uma estrutura de dados
#ChainedHashTable# usa \emph{hashing com encadeamento} para
guardar dados como um array #t# de listas. Um inteiro #n# registra o
número total de itens em todas as listas.
(ver a \figref{chainedhashtable}):
\codeimport{ods/ChainedHashTable.t.n}
\begin{figure}
   \begin{center}
     \includegraphics[width=\ScaleIfNeeded]{figs/chainedhashtable}
   \end{center}
   \caption[Uma ChainedHashTable]{Uma exemplo de #ChainedHashTable# com $#n#=14$ e $#t.length#=16$.  Nesse exemplo $#hash(x)#=6$}
   \figlabel{chainedhashtable}
\end{figure}
\index{valor hash}%
\index{hash(x)@#hash(x)#}%
O valor \emph{hash} de um item de dados #x#, denotado #hash(x)# é um valor
no intervalo 
$\{0,\ldots,#t.length#-1\}$.  Todos os itens com um valor hash #i#
são guardados na lista em 
#t[i]#.  Para garantir que as listas não ficam muito longas, mantemos a invariante 
\[
    #n# \le #t.length#
\]
tal que o número médio de elementos guardados em uma dessas listas é 
$#n#/#t.length# \le 1$.

Para adicionar um elemento #x# à tabela primeiro verificamos se o tamanho de #t# precisar ser aumentado e, se precisar, expandimos #t#.
Com isso resolvido, calculamos a hash de #x# para obter um inteiro #i# no
intervalo
$\{0,\ldots,#t.length#-1\}$ e adicionamos #x# à lista 
#t[i]#:
\codeimport{ods/ChainedHashTable.add(x)}
Expandir a tabela, caso necessário, envolve dobrar o comprimento de #t#
e reinserir todos os elementos na nova tabela. Essa 
estratégia é exatamente a mesma que aquela usada na implementação do
#ArrayStack# e o mesmo resultado se aplica.
O custo de expandir somente é visto como constante quando amortizado sobre uma sequência de
inserções (ver~o~\lemref{arraystack-amortized} na página~\pageref{lem:arraystack-amortized}).

Além da expansão, o único outro trabalho realizado ao adicionar um novo valor #x#
a uma #ChainedHashTable# envolve adicionar #x# à lista #t[hash(x)]#.  
Para qualquer das implementações de lista descrita nos Capítulos~\ref{chap:arrays}
ou \ref{chap:linkedlists}, isso leva somente um tempo constante.

Para remover um elemento #x# da tabela hash, iteramos sobre a lista
#t[hash(x)]# até acharmos #x# tal que podemos removê-lo:
\codeimport{ods/ChainedHashTable.remove(x)}
Isso leva $O(#n#_{#hash(x)#})$ de tempo, onde $#n#_{#i#}$ denota o comprimento da lista guardada em #t[i]#.

Buscar pelo elemento #x#
#x# em uma tabela hash é similar. Fazemos uma busca linear 
na lista #t[hash(x)]#:
\codeimport{ods/ChainedHashTable.find(x)}
Novamente, isso leva tempo proporcional ao comprimento da lista
 #t[hash(x)]#.

O desempenho de uma tabela hash depende criticamente na escolha da função 
 hash. Uma boa função hash irá espalhar os elementos de modo uniforme
 entre as #t.length# listas, tal que o tamanho esperado da lista 
#t[hash(x)]# é $O(#n#/#t.length)# = O(1)$.  Por outro lado,
uma função hash ruim irá distribuir todos os valores (incluindo #x#) à mesma posição da tabela. Nesse caso, o tamanho da lista #t[hash(x)]# será #n#.
Na seção a seguir descrevemos uma boa função hash.

\subsection{Hashing Multiplicativo}

\seclabel{multihash}

\index{multiplicativo!hashing}%
\index{hashing multiplicativo}%
Hashing multiplicativo é um método eficiente de gerar valores
hash baseado em aritmética modular 
 (discutido na \secref{arrayqueue})
 e divisão inteira.
Ele usa o operador $\ddiv$, que calcula a parte inteira de um quociente e
descarta o resto.
Formalmente, para quaisquer inteiros $a\ge 0$ e $b\ge 1$, $a\ddiv b = \lfloor
a/b\rfloor$.

Em hashing multiplicativo, usamos uma tabela hash de tamanho $2^{#d#}$ para um inteiro #d# (chamado de \emph{dimensão}).  A fórmula para aplicar hashing em um inteiro
$#x#\in\{0,\ldots,2^{#w#}-1\}$ é
\[
    #hash(x)# = ((#z#\cdot#x#) \bmod 2^{#w#}) \ddiv 2^{#w#-#d#} \enspace .
\]
Aqui, #z# é um inteiro ímpar escolhido aleatoriamente em 
$\{1,\ldots,2^{#w#}-1\}$.  Essa função hash pode ser computada muito eficientemente ao observar que, por padrão, operações em inteiros são 
módulo $2^{#w#}$ onde $#w#$ é o número de bits em um inteiro.\footnote{Isso é verdade para a maior parte de linguagens de programação incluindo
C, C\#, C++, e Java. Exceções notáveis são Python e 
Ruby, nas quais o resultado de uma operação inteira de tamanho fixo #w#-bit 
que passa do intervalo permitido é convertido a uma variável com tamanho de representação variável.} (Ver a \figref{multihashing}.) Além disso, 
divisão inteira por $2^{#w#-#d#}$
é equivalente a descartar os 
$#w#-#d#$ bits mais à direita em uma representação binária (
o que é implementado usando o deslocamento de bits $#w#-#d#$ à direita
usando o operador \javaonly{#>>>#}\cpponly{#>>#}\pcodeonly{#>>#}
).  \notpcode{Dessa forma, o código que implementa a fórmula acima é mais simples que a própria fórmula:}
\codeimport{ods/ChainedHashTable.hash(x)}

\begin{figure}
  \begin{center}
    \resizebox{.98\textwidth}{!}{
    \setlength{\arrayrulewidth}{1pt}
    \begin{tabular}{|lr@{}r|}\hline
    $2^#w#$ (4294967296)&            #1#&#00000000000000000000000000000000# \\
    #z# (4102541685)&                   &#11110100100001111101000101110101# \\
    #x# (42) &                          &#00000000000000000000000000101010# \\
    $#z#\cdot#x#$ &             #101000#&#00011110010010000101110100110010# \\
    $(#z#\cdot#x#)\bmod 2^{#w#}$ &      &#00011110010010000101110100110010# \\
    $((#z#\cdot#x#)\bmod 2^{#w#})\ddiv 2^{#w#-#d#}$ &&
                      \multicolumn{1}{@{}l|}{#00011110#} \\\hline
    \end{tabular}}
    \setlength{\arrayrulewidth}{.4pt}
  \end{center}
  \caption{A operação da função de hashing multiplicativo com $#w#=32$
   e $#d#=8$.}
  \figlabel{multihashing}
\end{figure}

O lema a seguir, cuja prova será feita posteriormente nesta seção,
mostra que o hashing multiplicativo faz um bom trabalho em evitar colisões: 

\begin{lem}\lemlabel{universal-hashing}
  Sejam #x# e #y# dois valores em $\{0,\ldots,2^{#w#}-1\}$ com 
  $#x#\neq #y#$. Então $\Pr\{#hash(x)#=#hash(y)#\} \le 2/2^{#d#}$.
\end{lem}

Com o
\lemref{universal-hashing}, o desempenho de #remove(x)# e 
#find(x)# são fáceis de analisar: 

\begin{lem}
  Para qualquer valor
   #x#, o comprimento esperado da lista #t[hash(x)]#
   é no máximo 
  $#n#_{#x#} + 2$, onde $#n#_{#x#}$ é o número de ocorrências de 
  #x# na tabela hash.
\end{lem}

\begin{proof}
  Seja $S$ o conjunto de elementos guardado na tabela hash 
  que não são iguais a #x#. Para um elemento 
  $#y#\in S$, defina a variável indicadora 
    \[ I_{#y#} = \left\{\begin{array}{ll}
       1 & \mbox{se $#hash(x)#=#hash(y)#$} \\
       0 & \mbox{caso contrário}
       \end{array}\right.
    \]
    e note que, segundo o  
 \lemref{universal-hashing}, $\E[I_{#y#}] \le
  2/2^{#d#}=2/#t.length#$.  O comprimento esperado da lista #t[hash(x)]#
  é dado por 
  \begin{eqnarray*}
   \E\left[#t[hash(x)].size()#\right] &=& \E\left[#n#_{#x#} + \sum_{#y#\in S} I_{#y#}\right] \\
    &=& #n#_{#x#} + \sum_{#y#\in S} \E [I_{#y#} ] \\
    &\le& #n#_{#x#} + \sum_{#y#\in S} 2/#t.length# \\
    &\le& #n#_{#x#} + \sum_{#y#\in S} 2/#n# \\
    &\le& #n#_{#x#} + (#n#-#n#_{#x#})2/#n# \\
    &\le& #n#_{#x#} + 2 \enspace ,
  \end{eqnarray*}
  conforme necessário. 
\end{proof}

Agora, queremos provar 
 o \lemref{universal-hashing}, mas primeiro precisamos de um resultado
 da Teoria de Números. Na prova a seguir, usamos a notação
$(b_r,\ldots,b_0)_2$ para denotar $\sum_{i=0}^r b_i2^i$, onde cada $b_i$
é um bit, 0 ou 1. Em outras palavras, 
$(b_r,\ldots,b_0)_2$ é o inteiro cuja representação binária é dada por
$b_r,\ldots,b_0$.
Usamos 
 $\star$ para denotar um valor de bit desconhecido. 

\begin{lem}\lemlabel{hashing-mapping}
  Seja $S$ o conjunto de inteiros ímpares em $\{1,\ldots,2^{#w#}-1\}$; seja $q$
  e $i$ quaisquer dois elementos em $S$.  Então há exatamente um valor $#z#\in S$ tal que $#z#q\bmod 2^{#w#} = i$.
\end{lem}

\begin{proof}
  Como o número de escolhas para 
   $#z#$ e $i$ é o mesmo, é suficiente provar que há \emph{no máximo} um valor
  $#z#\in S$ que satisfaz 
  $#z#q\bmod 2^{#w#} = i$.

  Suponha, para efeitos de uma contradição, que há dois tais valores
  #z# e #z'#, com $#z#>#z#'$. Portanto 
  \[
     #z#q\bmod 2^{#w#} = #z#'q \bmod 2^{#w#} = i
  \]
 Então
  \[ 
     (#z#-#z#')q\bmod 2^{#w#} = 0 
  \]
  Mas isso significa que
  \begin{equation}
    (#z#-#z#')q = k 2^{#w#} \eqlabel{factors} 
  \end{equation}
  para algum inteiro $k$.  Pensando em termos de números binários, temos 
  \[
    (#z#-#z#')q = k\cdot(1,\underbrace{0,\ldots,0}_{#w#})_2 \enspace ,
  \]
  tal que os últimos #w# bits na representação binária de 
  $(#z#-#z#')q$ são todos 0.

  Além disso, 
   $k\neq 0$, pois $q\neq 0$ e $#z#-#z#'\neq 0$.  Uma vez que $q$
   é ímpar, não há um 0 terminando sua representação binária:
  \[
    q = (\star,\ldots,\star,1)_2 \enspace .
  \]
  Como 
   $|#z#-#z#'| < 2^{#w#}$, $#z#-#z#'$ tem menos que #w# bits no final de sua representação binária:
  \[
    #z#-#z#' = (\star,\ldots,\star,1,\underbrace{0,\ldots,0}_{<#w#})_2
      \enspace .
  \]
  Portanto, o produto 
  $(#z#-#z#')q$ tem menos que #w# bits 0 no término da sua representação binária:
  \[
   (#z#-#z#')q = (\star,\cdots,\star,1,\underbrace{0,\ldots,0}_{<#w#})_2 
    \enspace .
  \]
  Portanto
   $(#z#-#z#')q$ não pode satisfazer a \myeqref{factors}, 
   chegando a uma contradição e completando a prova.
\end{proof}

A utilidade do 
 \lemref{hashing-mapping} vem da seguinte observação: 
 Se #z# for escolhido de forma uniformemente aleatória de $S$, então #zt#
 é uniformemente distribuído sobre $S$. Na prova a seguir,
 ajuda a lembrar da representação binária de #z#, que consiste de 
$#w#-1$ bits aleatórios seguidos por um 1.

\begin{proof}[Prova do \lemref{universal-hashing}]
  Primeiro notamos que a condição $#hash(x)#=#hash(y)#$ equivale à 
   afirmação os #d# bits de ordem mais alta em
  ``$#z# #x#\bmod2^{#w#}$ e os #d# bits de ordem mais alta de #z#
  $#z# #y#\bmod 2^{#w#}$ são os mesmos.''
  Uma condição necessária dessa afirmação é que os #d# bits de ordem mais
  alta na representação de $#z#(#x#-#y#)\bmod 2^{#w#}$
  são todos 0 ou todos 1. Isto é, 
  \begin{equation}
      #z#(#x#-#y#)\bmod 2^{#w#} = 
      (\underbrace{0,\ldots,0}_{#d#},\underbrace{\star,\ldots,\star}_{#w#-#d#})_2 
      \eqlabel{all-zeros}
  \end{equation}
  quando $#zx#\bmod 2^{#w#} > #zy#\bmod 2^{#w#}$ ou
  \begin{equation}
      #z#(#x#-#y#)\bmod 2^{#w#} = 
      (\underbrace{1,\ldots,1}_{#d#},\underbrace{\star,\ldots,\star}_{#w#-#d#})_2 
       \enspace .
      \eqlabel{all-ones}
  \end{equation}
  quando $#zx#\bmod 2^{#w#} < #zy#\bmod 2^{#w#}$.
  Portanto, só temos que delimitar a probabilidade de que 
  $#z#(#x#-#y#)\bmod 2^{#w#}$ pareça a \myeqref{all-zeros} ou \myeqref{all-ones}.
  
  Seja $q$ ser um inteiro único tal que $(#x#-#y#)\bmod
  2^{#w#}=q2^r$ para algum inteiro $r\ge 0$. Pelo
  \lemref{hashing-mapping}, a representação binária de $#z#q\bmod
  2^{#w#}$ tem $#w#-1$ bits aleatórios, seguidos por um 1: 
  \[
   #z#q\bmod 2^{#w#}  = (\underbrace{b_{#w#-1},\ldots,b_{1}}_{#w#-1},1)_2
  \]
  Portanto, a representação binária de 
   $#z#(#x#-#y#)\bmod 2^{#w#}=#z#q2^r\bmod 2^{#w#}$ tem
  $#w#-r-1$ bits aleatórios, seguidos por um 1, seguidos por $r$ dígitos 0:
  \[
  #z#(#x#-#y#)\bmod 2^{#w#}  =
  #z#q2^{r}\bmod 2^{#w#} =
      (\underbrace{b_{#w#-r-1},\ldots,b_{1}}_{#w#-r-1},1,\underbrace{0,0,\ldots,0}_{r})_2
  \]
  Podemos agora terminar a prova: se 
  $r > #w#-#d#$, então os #d# bits de alta ordem de 
  $#z#(#x#-#y#)\bmod 2^{#w#}$  contêm tanto 0 quanto 
  1, tal que a probabilidade de que $#z#(#x#-#y#)\bmod 2^{#w#}$ sejam 
  \myeqref{all-zeros} ou \myeqref{all-ones} é 0. 


  Se $#r#=#w#-#d#$, então a probabilidade de ocorrer 
  \myeqref{all-zeros} é 0, mas a probabilidade de ocorrer
  \myeqref{all-ones} é $1/2^{#d#-1}=2/2^{#d#}$
  (como precisamos ter $b_1,\ldots,b_{d-1}=1,\ldots,1$).  Se $r < #w#-#d#$,
  então precisamos ter
   $b_{#w#-r-1},\ldots,b_{#w#-r-#d#}=0,\ldots,0$ ou
  $b_{#w#-r-1},\ldots,b_{#w#-r-#d#}=1,\ldots,1$.  A probabilidade de cada um desses casos é 
  $1/2^{#d#}$ e eles são mutualmente exclusivos, então a probabilidade de cada desses casos é 
  $2/2^{#d#}$.  Isso completa a prova. 
\end{proof}

\subsection{Resumo}

O teorema a seguir resume o desempenho da estrutura de dados
#ChainedHashTable#:


\begin{thm}\thmlabel{hashtable}
  Uma #ChainedHashTable# implementa a interface #USet#. Ignorando o custo das chamadas a 
  #grow()#, uma #ChainedHashTable# possui as operações #add(x)#,
  #remove(x)# e #find(x)# em $O(1)$ de tempo esperado por operação. 

  Além disso, começando com uma 
  #ChainedHashTable# vazia, qualquer sequência de $m$ operações 
  #add(x)# e #remove(x)# resulta em um total de $O(m)$ de tempo gasto durante
  todas a chamadas a #grow()#.
\end{thm}

\section{#LinearHashTable#: Sondagem Linear}
\seclabel{linearhashtable}

\index{LinearHashTable@#LinearHashTable#}%
A estrutura de dados #ChainedHashTable# usa um array de listas onde a #i#-ésima lista guarda todos os elementos #x# tais que 
$#hash(x)#=#i#$.  Uma outra opção de projeto, chamado de \emph{endereçamento aberto}, 
\index{endereçamento aberto}%
é guardar elementos diretamente em um array #t# sendo cada posição #t# para armazenar no máximo um valor.
Essa abordagem é implementada pela 
#LinearHashTable# descrita nesta seção.
Em alguns lugares, essa estrutura de dados é descrita como \emph{endereçamento aberto com sondagem linear}.
\index{sondagem linear}%

A principal ideia por trás da
#LinearHashTable# é que deveríamos, idealmente,
guardar o elemento #x# com valor hash 
#i=hash(x)# na posição da tabela
#t[i]#.  Se não podemos fazer isso (porque algum elemento já está armazenado lá)
então tentamos armazená-lo na posição
 $#t#[(#i#+1)\bmod#t.length#]$;
se isso não for possível, então tentamos
$#t#[(#i#+2)\bmod#t.length#]$,
e assim por diante, até encontrarmos um lugar para #x#.

Há três tipos de entradas guardadas em #t#:
\begin{enumerate}
  \item valores de dados: valores reais no #USet# que estamos representando; 
  \item valores #null#: em posições do array onde nenhum valor foi guardado antes; e
  \item valores #del#: em posições do array onde dados foram previamente guardados mas que desde então foram deletados. 
\end{enumerate}
Em adição ao contador 
 #n# que registra o número de elementos na 
#LinearHashTable#, um contador, #q#, registra o número de elementos dos Tipos~1 e 3.
Isso é, #q# é igual a #n# mais o número de valores #del# em #t#. Para fazer isso eficientemente, precisamos que #t# seja consideravelmente maior que #q#, tal que haja muitos #null# em #t#.
As operações em 
#LinearHashTable# portanto mantém a invariante
$#t.length#\ge 2#q#$.

Para resumir, uma
#LinearHashTable# contém um array #t# que guarda elementos de dados 
e inteiros
#n# e #q# que registram o número de elementos de dados e de valores não-#null# de #t# respectivamente. Como muitas funções hash somente funcionam para tamanhos de tabela que são potência de 2, também mantemos um inteiro #d# e a invariante 
$#t.length#=2^#d#$.
\codeimport{ods/LinearHashTable.t.n.q.d}

A operação #find(x)# na #LinearHashTable# é simples. 
Iniciamos na posição do array #t[i]# onde
$#i#=#hash(x)#$ e buscamos as posições #t[i]#,
$#t#[(#i#+1)\bmod #t.length#]$, $#t#[(#i#+2)\bmod #t.length#]$, e assim segue,
até encontrarmos um índice #i'# tal que 
 #t[i']=x# ou #t[i']=null#.
No primeiro caso retornamos
 #t[i']#. No outro, quando #t[i']=null#, concluímos
 que #x# não está contido na tabela hash e retornamos #null#.
\codeimport{ods/LinearHashTable.find(x)}

A operação #add(x)# também é razoavelmente simples para implementar. Após verificar 
que #x# não está na tabela (usando #find(x)#), procuramos em 
#t[i]#, $#t#[(#i#+1)\bmod #t.length#]$, $#t#[(#i#+2)\bmod #t.length#]$,
e assim por diante, até acharmos um 
#null# ou #del# e guardamos #x# naquela posição, 
incrementando #n# e #q#, se apropriado.
\codeimport{ods/LinearHashTable.add(x)}

Até agora, a implementação da operação #remove(x)# deve ser óbvia.
Buscamos
 #t[i]#, $#t#[(#i#+1)\bmod #t.length#]$, $#t#[(#i#+2)\bmod
#t.length#]$, e assim por diante até acharmos um índice #i'# tal que #t[i']=x#
ou #t[i']=null#.  No primeiro caso, atribuímos #t[i']=del# e retornamos
#true#.  No segundo caso, concluímos que #x# não foi armazenado na tabela 
(e portanto não pode ser deletado) e retornamos #false#.
\codeimport{ods/LinearHashTable.remove(x)}

A corretude dos métodos
#find(x)#, #add(x)# e #remove(x)# é fácil de verificar embora dependa do uso dos valores #del#. Note que nenhuma dessas operações nunca atribui #null# a uma posição diferente de #null#.
Portanto, se alcançarmos um índice #i'# tal que 
 #t[i']=null# isso comprova que o elemento
 #x# pelo qual buscamos não está guardado na tabela
; #t[i']# sempre foi #null#, então não há porque uma operação prévia #add(x)#
continuar além do índice #i'#.

O método 
#resize()# é chamado por #add(x)# quando o número de posições não-#null#
exceder $#t.length#/2$ ou por #remove(x)# quando o número de entrada de dados 
for bem menor que #t.length/8#.  O método #resize()# funciona como os métodos 
#resize()# em outras estruturas de dados baseadas em array.
Achamos o menor inteiro não negativo #d# tal que 
$2^{#d#} \ge 3#n#$.  Realocamos o array #t# tal que tenha tamanho $2^{#d#}$,
e então inserimos todos os elementos na versão antiga de #t# na nova
cópia redimensionada de #t#. Ao fazer isso, reiniciamos #q# igual a #n#
pois o novo #t# contém nenhum valor #del#.
\codeimport{ods/LinearHashTable.resize()}

\subsection{Análise da Sondagem Linear}
Note que cada operação
#add(x)#, #remove(x)# ou #find(x)#, termina assim que (ou mesmo antes)  
encontra a primeira posição #null# em #t#.
A intuição por trás da análise da sondagem linear é que, 
como pelo menos metade dos elementos em #t# são iguais a #null#, uma
operação não deve demorar muito para completar pois irá rapidamente achar
uma posição com valor #null#.
Não devemos usar essa intuição ao pé da letra pois levaria à conclusão (incorreta) de que o número esperado de posições em #t# que foram examinadas por uma operação
é de até 2.

Para o resto desta seção, iremos assumir que todos os valores hash
são independentemente e uniformemente distribuídos em 
$\{0,\ldots,#t.length#-1\}$.
Essa não é uma premissa realista, mas permitirá uma análise da sondagem linear.
Mais a seguir, nesta seção, iremos descrever um método, chamado 
de hashing de tabulação, que produz uma função hash que é 
``boa o suficiente'' para sondagem linear. 
Também isso assumir que todos os índices das posições de #t#
são obtidos módulo #t.length#, tal que #t[i]# é na verdade um atalho para 
$#t#[#i#\bmod#t.length#]$.

\index{run}%
Dizemos que um \emph{agrupamento de comprimento $k$ iniciando em #i#} ocorre
quando as posições da tabela
$#t[i]#, #t[i+1]#,\ldots,#t#[#i#+k-1]$ não são #null#
e $#t#[#i#-1]=#t#[#i#+k]=#null#$. O número de elementos não-#null# de 
#t# é exatamente #q# e o método #add(x)# garante que, a qualquer momento, 
$#q#\le#t.length#/2$.  Há #q# elementos $#x#_1,\ldots,#x#_{#q#}$
que foram inseridos em #t# desde a última operação #resize()#.
De acordo com nossa premissa, cada um desses elementos tem valor hash
 $#hash#(#x#_j)$ que é de uma distribuição uniforme e cada qual é independente do resto.
Nesse cenário podemos provar o principal lema necessário para analisar
a sondagem linear. 

\begin{lem}\lemlabel{linear-probing}
Fixe um valor $#i#\in\{0,\ldots,#t.length#-1\}$.  Então a probabilidade de que um agrupamento de 
comprimento $k$ inicie-se em #i# é $O(c^k)$ para uma constante $0<c<1$.
\end{lem}

\begin{proof}
Se um agrupamento de comprimento $k$ se iniciar em #i#, então haverá exatamente #k#
elementos 
$#x#_j$ tais que $#hash#(#x#_j)\in\{#i#,\ldots,#i#+k-1\}$.
A probabilidade disso ocorrer é exatamente 
\[
  p_k  = \binom{#q#}{k}\left(\frac{k}{#t.length#}\right)^k\left(\frac{#t.length#-k}{#t.length#}\right)^{#q#-k} \enspace ,
\]
pois para alguma escolha de $k$ elementos, eles precisam mapear 
com a hash para uma dessas $k$ posições e o restante dos  
$#q#-k$ elementos precisa mapear para as outras 
 $#t.length#-k$ posições na tabela.

\footnote{Note que $p_k$ é maior que a probabilidade de que
  um agrupamento de tamanho $k$ inicie em #i#, pois a definição de #p_k# não inclui o requisito $#t#[#i#-1]=#t#[#i#+k]=#null#$.}

Na próxima derivação iremos trapacear um pouco e substituir $r!$ com 
$(r/e)^r$. A aproximação de Stirling (\secref{factorials}) mostra que
  isso difere apenas
$O(\sqrt{r})$ da verdade. Fazemos isso para simplificar a derivação; 
o \excref{linear-probing} pede ao leitor refazer os cálculos de maneira mais
  rigorosa usando a aproximação de Stirling completa.

  O valor de $p_k$ é maximizado quando #t.length# é mínimo 
  e a estrutura de dados mantém o invariante 
$#t.length# \ge 2#q#$, então
\begin{align*}
   p_k & \le \binom{#q#}{k}\left(\frac{k}{2#q#}\right)^k\left(\frac{2#q#-k}{2#q#}\right)^{#q#-k} \\
  & = \left(\frac{#q#!}{(#q#-k)!k!}\right)\left(\frac{k}{2#q#}\right)^k\left(\frac{2#q#-k}{2#q#}\right)^{#q#-k} \\
  & \approx \left(\frac{#q#^{#q#}}{(#q#-k)^{#q#-k}k^k}\right)\left(\frac{k}{2#q#}\right)^k\left(\frac{2#q#-k}{2#q#}\right)^{#q#-k} && \text{[Aproximação de Stirling]} \\
  & = \left(\frac{#q#^{k}#q#^{#q#-k}}{(#q#-k)^{#q#-k}k^k}\right)\left(\frac{k}{2#q#}\right)^k\left(\frac{2#q#-k}{2#q#}\right)^{#q#-k} \\
 & = \left(\frac{#q#k}{2#q#k}\right)^k
     \left(\frac{#q#(2#q#-k)}{2#q#(#q#-k)}\right)^{#q#-k} \\
 & = \left(\frac{1}{2}\right)^k
     \left(\frac{(2#q#-k)}{2(#q#-k)}\right)^{#q#-k} \\
 & = \left(\frac{1}{2}\right)^k
     \left(1+\frac{k}{2(#q#-k)}\right)^{#q#-k} \\
 & \le \left(\frac{\sqrt{e}}{2}\right)^k \enspace .
\end{align*}
(No último passo, usamos a desigualdade $(1+1/x)^x \le e$, que vale para
  todo $x>0$.)  Como $\sqrt{e}/{2}< 0.824360636 < 1$, isso completa a prova. 
\end{proof}

Usar o \lemref{linear-probing} para provar limitantes superiores no tempo 
de execução esperado de #find(x)#, #add(x)# e #remove(x)# agora é 
mais simples.  Considere o caso mais simples, onde executamos 
#find(x)# para algum valor #x# nunca foi guardado em #LinearHashTable#.  
Nesse caso, $#i#=#hash(x)#$ é uma variável aleatória em 
$\{0,\ldots,#t.length#-1\}$ independente do conteúdo de #t#.  Se #i#
é parte de um agrupamento de tamanho $k$, então o tempo que leva para executar
a operação #find(x)# é até $O(1+k)$.  Então, o tempo esperado de execução pode ser limitado superiormente por 
\[
  O\left(1 + \left(\frac{1}{#t.length#}\right)\sum_{i=1}^{#t.length#}\sum_{k=0}^{\infty} k\Pr\{\text{#i# é parte de um agrupamento de tamanho $k$}\}\right) \enspace .
\]

Note que cada agrupamento de tamanho $k$ contribui à soma interna $k$ vezes para uma contribuição total de $k^2$, então a soma anterior pode ser reescrita como 
\begin{align*}
  & { } O\left(1 + \left(\frac{1}{#t.length#}\right)\sum_{i=1}^{#t.length#}\sum_{k=0}^{\infty} k^2\Pr\{\mbox{#i# inicia um agrupamento de tamanho $k$}\}\right) \\
  & \le O\left(1 + \left(\frac{1}{#t.length#}\right)\sum_{i=1}^{#t.length#}\sum_{k=0}^{\infty} k^2p_k\right) \\
  & = O\left(1 + \sum_{k=0}^{\infty} k^2p_k\right) \\
  & = O\left(1 + \sum_{k=0}^{\infty} k^2\cdot O(c^k)\right) \\
  & = O(1) \enspace .
\end{align*}
O último passo dessa derivação vem do fato que 
$\sum_{k=0}^{\infty} k^2\cdot O(c^k)$ é uma série exponencial decrescente.\footnote{Em muitos livros de cálculo, essa soma passa no teste de convergência:
existe um inteiro positivo $k_0$ tal que, para todo 
 $k\ge k_0$, $\frac{(k+1)^2c^{k+1}}{k^2c^k} < 1$.}

 Portanto, concluímos que o tempo de execução esperado da operação #find(x)#
 para um valor #x# que não é contido em uma 
 #LinearHashTable# é $O(1)$.

 Se ignorarmos o custo da operação
#resize()# , então a análise acima nos dá tudo o que precisamos para analisar
o custo de operações em uma #LinearHashTable#.

Primeiramente, a análise de #find(x)# dada acima aplica-se à operação 
#add(x)# quando#x# não está na tabela. Para analisar a operação
 #find(x)# quando #x# está na tabela, precisamos somente notar
 que isso tem o mesmo custo da operação #add(x)# que anteriormente adicionou 
#x# à tabela.  Finalmente, o custo de uma operação #remove(x)#
é o mesmo que o custo de uma operação #find(x)#.

Em resumo, se ignorarmos o custo de chamadas a 
 #resize()#, todas as operações em uma 
#LinearHashTable# rodam em $O(1)$ de tempo esperado. Considerar o
custo de redimensionamento pode ser feito usando o mesmo tipo de análise amortizada
realizada para a estrutura de dados 
 #ArrayStack# feita na \secref{arraystack}.

\subsection{Resumo}

O seguinte teorema resume o desempenho da estrutura de dados
#LinearHashTable#:


\begin{thm}\thmlabel{linear-probing}
  Uma #LinearHashTable# implementa a interface #USet#. Ignorando o custo
  de chamadas a 
   #resize()#, uma #LinearHashTable# implementa as operações
  #add(x)#, #remove(x)# e #find(x)# em $O(1)$ de tempo esperado por 
  operação.  

  Além disso, ao começar com uma 
  #LinearHashTable#, qualquer sequência 
  de $m$ operações #add(x)# e #remove(x)# resulta em um total de 
  $O(m)$ de tempo gasto durante todas as chamadas a #resize()#.
\end{thm}

\subsection{Hashing por tabulação}
\seclabel{tabulation}

\index{hashing por tabulação}%
Ao analisarmos a estrutura #LinearHashTable#, fazemos uma suposição muito forte:
 para qualquer conjunto de elementos
 $\{#x#_1,\ldots,#x#_#n#\}$ os valores de hash
$#hash#($x$_1),\ldots,#hash#(#x#_#n#)$ são independentes e
uniformemente distribuídos sobre o conjunto 
 $\{0,\ldots,#t.length#-1\}$.  Um jeito de obter isso é guardar em um array gigante #tab# de tamanho 
$2^{#w#}$, onde cada entrada é um inteiro aleatório #w#-bit, independente de todas
as outras posições. Dessa maneira, poderíamos implementar #hash(x)# pela
extração de um inteiro com #d# bits da #tab[x.hashCode()]#:
\codeimport{ods/LinearHashTable.idealHash(x)}
\pcodeonly{Aqui, #>>#, é o operador de \emph{deslocamento bit-a-bit à direita}, então 
#x.hashCode() >> w-d# extrai os #d# bits mais significativos do código hash dos #w# bits do código hash de #x#}

Infelizmente, guardar um array de tamanho 
 $2^{#w#}$ é proibitivo em termos de uso de memória. 
A abordagem usada pela \emph{hashing por tabulação} é, em vez disso, tratar 
inteiros de #w# bits como sendo compostos de 
$#w#/#r#$ inteiros, cada qual tem somente $#r#$ bits.  
Dessa maneira, hashing por tabulação precisa somente de 
$#w#/#r#$ arrays cada qual com tamanho $2^{#r#}$.  Todas as entradas nesses arrays são inteiros aleatórios independentes com #w# bits. 

Para obter o valor de 
#hash(x)# dividimos #x.hashCode()# em $#w#/#r#$ inteiros com #r# bits
e os usamos como índices desses arrays. Então combinamos 
todos esses valores com o operador bit-a-bit ou-exclusivo para obter #hash(x)#.
O código a seguir mostra como isso funciona quando 
$#w#=32$ e $#r#=4$:
\codeimport{ods/LinearHashTable.hash(x)}
Nesse caso,
 #tab# é um array bidimensional com quatro colunas e 
$2^{32/4}=256$ linhas.
\pcodeonly{
  Quantidades como 
#0xff#, usadas acima, são \emph{números hexadecimais}
\index{números hexadecimais}%
cujos dígitos tem 16 possíveis valores 0--9, com o significado usual e
a--f, que denotam valores de 10 a 15. O número
$#0xff#=15\cdot 16 + 15 = 255$.
O símbolo
#&# é o operador \emph{bit-a-bit and} então, o código
#h >> 8 & 0xff# extrai os bits com índices de 8 a 15 de #h#.}

É fácil verificar que, para qualquer #x#, #hash(x)# é uniformemente
distribuído sobre $\{0,\ldots,2^{#d#}-1\}$.  Com um pouco de trabalho,
é possível verificar que qualquer par de valores tem suas hash independentes.
Isso implica que hashing por tabulação poderia ser usado no lugar
do hashing multiplicativo para a implementação da #ChainedHashTable#.

Entretanto, não é verdade que qualquer conjunto de #n# valores distintos
resulte em #n# valores hash independentes. De qualquer forma,
quando hashing por tabulação é usado, o limitante 
\thmref{linear-probing} ainda vale. Referências para isso 
são providas no final deste capítulo. 

\section{Códigos Hash}
\index{código hash}%
Tabelas hash discutidas na seção anterior são usadas para associar
dados com chaves inteiras consistindo de #w# bits. Em muitos casos,
temos chaves que não são inteiros. Elas podem ser strings, objetos, arrays,
ou outras estruturas compostas. Para usar tabelas hash para esses tipos de dados,
precisamos mapear esses tipos de dados para códigos hash de #w# bits.
Mapeamentos de códigos hash devem ter as seguintes propriedades:

\begin{enumerate}
  \item Se #x# e #y# são iguais, então #x.hashCode()# e #y.hashCode()# são iguais.
  \item Se #x# e #y# não são iguais, então a probabilidade que 
  $#x.hashCode()#=#y.hashCode()#$ deve ser pequena (próxima a $1/2^{#w#}$).
\end{enumerate}

A primeira propriedade assegura que se guardarmos #x# em um tabela hash
e depois olharmos um valor #y# igual a #x#, então iremos achar #x# --- conforme deveríamos.
A segunda propriedade minimiza a perda ao converter nossos objetos a inteiros.
Ela garante que objetos distintos usualmente tenham códigos hash diferentes
e então provavelmente são guardados em posições distintas na nossa
tabela hash.

\subsection{Códigos hash para Tipos de Dados Primitivos}
\index{código hash!para dados primitivos}%

Tipos de dados primitivos pequenos como 
#char#, #byte#, #int# e #float# são normal fáceis de obter códigos hash.
Esses tipos de dados sempre têm uma representação binária que usualmente consistem de #w# ou menos bits. 
\javaonly{(Por exemplo, em Java, #byte# é um tipo de 8 bits
e #float# é um tipo de 32 bits.)}\cpponly{(Por exemplo, em C++ #char#
é tipicamente de 8 bits e #float# é um tipo de 32 bits.)} Nesses casos,
nós somente tratamos esses bits como a representação de um inteiro no
intervalo $\{0,\ldots,2^#w#-1\}$.  Se dois valores são diferentes, eles recebem códigos hash distintos. Se são iguais, recebem o mesmo código hash. 

Alguns dados de tipos primitivos são compostos de mais de #w# bits, normalmente
$c#w#$ bits para alguma constante $c$. (No Java, os tipos #long# e #double#
são exemplos disso com $c=2$.)  Esses tipos de dados podem ser tratados como objetos compostos de $c$ partes, conforme descrito na seção a seguir. 

\subsection{Códigos Hash para Objetos Compostos}
\seclabel{stringhash}
\index{códigos hash!para objetos compostos}%
Para um objeto composto, queremos criar um código hash combinando os códigos
hash individuais das partes que constituem o objeto.
Isso não é tão fácil quanto parece.

Embora seja possível achar muitos hacks para isso (por exemplo, combinar os códigos hash com operações bitwise ou-exclusivo), muitos desses hacks acabam sendo fáceis de enganar (veja os Exercícios~\ref{exc:hash-hack-first}--\ref{exc:hash-hack-last}).
Porém, se pudermos fazer aritmética com $2#w#$ bits de precisão, então
existem métodos simples e robustos a nossa disposição.
Suponha que temos um objeto composto de várias partes
$P_0,\ldots,P_{r-1}$ cujos códigos hash são $#x#_0,\ldots,#x#_{r-1}$.
Então, podemos escolher inteiros de #w# bits mutuamente independentes 
$#z#_0,\ldots,#z#_{r-1}$ e um inteiro ímpar aleatório com $2#w#$ bits #z# 
e computar um código hash para nosso objeto com 
\[
   h(#x#_0,\ldots,#x#_{r-1}) =  
   \left(\left(#z#\sum_{i=0}^{r-1} #z#_i #x#_i\right)\bmod 2^{2#w#}\right)
   \ddiv 2^{#w#} \enspace .
\]
Note que esse código hash tem um passo final (multiplicar por #z# e dividir
por $2^{#w#}$) que usa uma função hash multiplicativo da \secref{multihash} para obter um resultado intermediário $2#w#$ bits 
e reduzir a um resultado final de #w# bits. Aqui segue um exemplo desse método
aplicado a um único objeto composto de três partes
 #x0#, #x1# e #x2#:
\javaimport{junk/Point3D.x0.hashCode()}
\cppimport{ods/Point3D.hashCode()}
\pcodeimport{ods/Point3d.hashCode()}

O teorema a seguir mostra que, em adição a ser de simples implementação, esse
método é comprovadamente bom:

\begin{thm}\thmlabel{multihash}
Sejam $#x#_0,\ldots,#x#_{r-1}$ e $#y#_0,\ldots,#y#_{r-1}$ sequências de inteiros de #w# bits em $\{0,\ldots,2^{#w#}-1\}$ e assuma $#x#_i \neq #y#_i$ para pelo menos um índice $i\in\{0,\ldots,r-1\}$. Então 
\[
   \Pr\{ h(#x#_0,\ldots,#x#_{r-1}) =  h(#y#_0,\ldots,#y#_{r-1}) \} 
        \le 3/2^{#w#} \enspace .  
\] 
\end{thm}

\begin{proof}
  Iremos, a princípio, ignorar o passo final de hashing multiplicativo 
  e ver como aquele passo contribui depois. Defina:
  \[
    h'(#x#_0,\ldots,#x#_{r-1}) =  
       \left(\sum_{j=0}^{r-1} #z#_j #x#_j\right)\bmod 2^{2#w#} \enspace .
  \]
  Assuma que $h'(#x#_0,\ldots,#x#_{r-1}) =  h'(#y#_0,\ldots,#y#_{r-1})$.
  Podemos reescrever isso como: 
  \begin{equation}  \eqlabel{bighash-x}
      #z#_i(#x#_i-#y#_i) \bmod 2^{2#w#} = t
  \end{equation}
  onde 
  \[
     t = \left(\sum_{j=0}^{i-1} #z#_j(#y#_j-#x#_j) + \sum_{j=i+1}^{r-1} #z#_j(#y#_j-#x#_j)\right) \bmod 2^{2#w#}
  \]
  Se assumirmos, sem perda de generalidade, que 
  $#x#_i> #y#_i$, então
  a \myeqref{bighash-x} torna-se
  \begin{equation}
      #z#_i(#x#_i-#y#_i) = t \eqlabel{bighash-xx} \enspace ,
  \end{equation}
  pois cada 
   $#z#_i$ e $(#x#_i-#y#_i)$ são até $2^{#w#}-1$, então os seus produtos
   são até 
  $2^{2#w#}-2^{#w#+1}+1 < 2^{2#w#}-1$.

  Por suposição,
  $#x#_i-#y#_i\neq 0$, então a \myeqref{bighash-xx} tem no máximo uma solução
  em $#z#_i$.  Portanto, como $#z#_i$ e $t$ são 
  independentes ($#z#_0,\ldots,#z#_{r-1}$ são mutuamente independentes), a probabilidade de selecionarmos 
  $#z#_i$
  tal que $h'(#x#_0,\ldots,#x#_{r-1})=h'(#y#_0,\ldots,#y#_{r-1})$ é no máximo 
  $1/2^{#w#}$.

  O passo final da função hash é aplicar o hashing multiplicativo
  para reduzir nosso resultado intermediário de 
  $2#w#$ bits $h'(#x#_0,\ldots,#x#_{r-1})$ a um
  resultado final com #w# bits $h(#x#_0,\ldots,#x#_{r-1})$.  Usando o \thmref{multihash},
  se $h'(#x#_0,\ldots,#x#_{r-1})\neq h'(#y#_0,\ldots,#y#_{r-1})$, então
  $\Pr\{h(#x#_0,\ldots,#x#_{r-1}) = h(#y#_0,\ldots,#y#_{r-1})\} \le 2/2^{#w#}$.

  Para resumir,
  \begin{align*}
    & \Pr\left\{\begin{array}{l}
          h(#x#_0,\ldots,#x#_{r-1}) \\
          \quad = h(#y#_0,\ldots,#y#_{r-1})\end{array}\right\} \\
      &= \Pr\left\{\begin{array}{ll}
            \mbox{$h'(#x#_0,\ldots,#x#_{r-1}) = h'(#y#_0,\ldots,#y#_{r-1})$ ou} \\
            \mbox{$h'(#x#_0,\ldots,#x#_{r-1}) \neq h'(#y#_0,\ldots,#y#_{r-1})$} \\
                 \quad  \mbox{e
$#z#h'(#x#_0,\ldots,#x#_{r-1})\ddiv2^{#w#} = #z# h'(#y#_0,\ldots,#y#_{r-1})\ddiv 2^{#w#}$}
          \end{array}\right\} \\
      &\le 1/2^{#w#} + 2/2^{#w#} = 3/2^{#w#} \enspace . \qedhere
  \end{align*}
\end{proof}

\index{código hash!para strings}%
\index{código hash!para arrays}%
\subsection{Códigos Hash para Arrays e Strings}
\seclabel{polyhash}

O método da seção anterior funciona bem para objetos que tem um número constante, fixo, de componentes. Entretanto, ele não funciona bem quando queremos usá-lo com
objetos que tem um número variável de componentes,
pois querer um inteiro aleatório de #w# bits 
$#z#_i$ para cada componente.
Poderíamos usar uma sequência pseudo aleatória para gerar quantos 
$#z#_i$ fossem necessários, mas então os $#z#_i$ não seriam mutuamente independentes e se torna difícil provar que os números pseudo aleatórios não interagem mal com a função hash que estamos usando. 
Em particular, os valores de $t$ e 
$#z#_i$ na prova do \thmref{multihash} não seriam mais independentes.

\index{prime field}%
\index{corpo primal}%
Uma abordagem mais rigorosa é obter nossos códigos hash com base em polinômios sobre corpos primais; esse são basicamente polinômios comuns que são avaliados em módulo algum número primo, #p#. Esse método é baseado no seguinte teorema, que
diz que polinômios sobre corpos primais se comportam como polinômios comuns:

\begin{thm}\thmlabel{prime-polynomial}
 Seja $#p#$ um número primo e seja $f(#z#) = #x#_0#z#^0 + #x#_1#z#^1 +
 \cdots + #x#_{r-1}#z#^{r-1}$ um polinômio não trivial com coeficientes 
 $#x#_i\in\{0,\ldots,#p#-1\}$. Então a equação $f(#z#)\bmod #p# = 0$
 tem até $r-1$ soluções para $#z#\in\{0,\ldots,p-1\}$.
\end{thm}

Para usar o \thmref{prime-polynomial}, aplicamos uma função hash em uma sequência de inteiros 
$#x#_0,\ldots,#x#_{r-1}$ cada qual com $#x#_i\in \{0,\ldots,#p#-2\}$ usando um 
inteiro aleatório 
$#z#\in\{0,\ldots,#p#-1\}$ via a fórmula
\[
   h(#x#_0,\ldots,#x#_{r-1}) 
    = \left(#x#_0#z#^0+\cdots+#x#_{r-1}#z#^{r-1}+(#p#-1)#z#^r \right)\bmod #p# \enspace .
\]

Note o termo extra 
$(#p#-1)#z#^r$ no fim da fórmula. Ajuda a pensar de 
$(#p#-1)$ como o último elemento, $#x#_r$, na sequência
$#x#_0,\ldots,#x#_{r}$.  Note que esse elemento difere de todos os outros elementos na sequência (cada qual está no conjunto 
$\{0,\ldots,#p#-2\}$).
Podemos pensar que $#p#-1$ atua como um marcador de fim de sequência.

O teorema a seguir, que considera o caso de duas sequências de mesmo 
tamanho, mostra que essa função hash produz bons resultados
para uma pequena quantidade de randomização necessária para escolher #z#:

\begin{thm}\thmlabel{stringhash-eqlen}
  Seja $#p#>2^{#w#}+1$ um primo, com $#x#_0,\ldots,#x#_{r-1}$ e 
  $#y#_0,\ldots,#y#_{r-1}$ sendo sequências de inteiros com #w# bits em 
  $\{0,\ldots,2^{#w#}-1\}$, e assuma $#x#_i \neq #y#_i$ para pelo menos 
  um índice $i\in\{0,\ldots,r-1\}$. Então
  \[
     \Pr\{ h(#x#_0,\ldots,#x#_{r-1}) =  h(#y#_0,\ldots,#y#_{r-1}) \} 
          \le (r-1)/#p# \enspace .
  \] 
\end{thm}

\begin{proof}
A equação $h(#x#_0,\ldots,#x#_{r-1}) =  h(#y#_0,\ldots,#y#_{r-1})$
pode ser reescrita como 
  \begin{equation}  \eqlabel{strhash-eqlen}
    \left(
       (#x#_0-#y#_0)#z#^0+\cdots+(#x#_{r-1}-#y#_{r-1})#z#^{r-1} 
    \right)\bmod #p# = 0.
  \end{equation}
Como $#x#_#i#\neq #y#_#i#$, esse polinômio é não-trivial. Portanto, usando o
  \thmref{prime-polynomial}, ele tem no máximo $r-1$ soluções em #z#.
  A probabilidade que escolhemos #z# para seja uma dessas soluções é portanto
  no máximo $(r-1)/#p#$.
\end{proof}

Note que essa função hash também considera o caso em que duas sequências tem comprimentos distintos, mesmo quando uma das sequências é um prefixo de outra.
Isso porque essa função efetivamente obtém o hash da sequência infinita
\[
  #x#_0,\ldots,#x#_{r-1}, #p#-1,0,0,\ldots \enspace .
\]
Isso garante que, se temos duas sequências de comprimento
 $r$ e $r'$
com $r > r'$, então essas duas sequências diferem no índice $i=r$.  
Nesse caso, a \myeqref{strhash-eqlen} torna-se
\[
  \left(
     \sum_{i=0}^{i=r'-1}(#x#_i-#y#_i)#z#^i + (#x#_{r'} - #p# + 1)#z#^{r'}
     +\sum_{i=r'+1}^{i=r-1}#x#_i#z#^i + (#p#-1)#z#^{r}
  \right)\bmod #p# = 0 \enspace ,
\]
que, segundo o \thmref{prime-polynomial}, tem até $r$ soluções em $#z#$.
Isso combinado com o 
 \thmref{stringhash-eqlen} é suficiente para provar o seguinte teorema mais geral:

\begin{thm}\thmlabel{stringhash}
Seja $#p#>2^{#w#}+1$ um número primo, usando duas sequências distintas de #w# bits $#x#_0,\ldots,#x#_{r-1}$ e
  $#y#_0,\ldots,#y#_{r'-1}$ em
  $\{0,\ldots,2^{#w#}-1\}$. Então
  \[
     \Pr\{ h(#x#_0,\ldots,#x#_{r-1}) =  h(#y#_0,\ldots,#y#_{r-1}) \} 
          \le \max\{r,r'\}/#p#  \enspace .  
  \] 
\end{thm}

O trecho de código a seguir mostra como essa função hash é aplicada a
um objeto que contém um array, #x#, de valores:
\javaimport{junk/GeomVector.hashCode()}
\cppimport{ods/GeomVector.hashCode()}
\pcodeimport{ods/GeomVector.hashCode()}

O código precedente  sacrifica alguma probabilidade de colisão por conveniência
na implementação. Em particular, ele aplica a função hash multiplicativa de 
\secref{multihash}, com $#d#=31$ para reduzir #x[i].hashCode()# a um valor de 31 bits.
Isso é feito dessa para que adições e multiplicações 
feitas módulo o primo
$#p#=2^{32}-5$ possam ser realizadas usando aritmética 63 bits sem sinal.
Então a probabilidade de duas diferentes sequências, a mais longa com comprimento #r#, 
terem o mesmo código hash é até
\[
    2/2^{31} + r/(2^{32}-5)
\]
em vez da 
$r/(2^{32}-5)$ especificada no \thmref{stringhash}.

\section{Discussão e Exercícios}

Tabelas hash e códigos hash representam um campo de pesquisa muito ativo 
que é brevemente pincelado neste capítulo. 
A online \emph{Bibliography on Hashing} \cite{hashing}
\index{Bibliography on Hashing}%
contém aproximadamente 2000 referências.

Existe uma grande variedade de implementações de diferentes tabelas.
Aquela descrita na 
\secref{hashtable} é conhecida como \emph{hashing with chaining} ou \emph{hashing com encadeamento}
\index{hashing with chaining}%
\index{hashing com encadeamento} 
(cada posição do array contém uma lista, por vezes encadeada, (#List#) de elementos).
Hashing com encadeamento tem suas origens em um
memorando interno da IBM escrito por 
H. P. Luhn que data de Janeiro de 1953. Esse memorando também parece
ser uma das primeiras referências a listas ligadas.

\index{open addressing}%
\index{endereçamento aberto}%
Uma alternativa a hashing com encadeamento é aquela que usa esquemas
de \emph{open addressing}, ou , \emph{endereçamento aberto}, onde todos 
os dados são armazenados diretamente em um array. 
Esses esquemas incluem a estrutura 
 #LinearHashTable# da 
\secref{linearhashtable}. Essa ideia também foi proposta independentemente por um
grupo na IBM na década de 1950. Esquemas de endereçamento aberto devem 
tratar do problema de resolução de colisões:
\emph{resolução de colisões}: 
\index{resolução de colisões}%
o caso em que dois valores hash mapeiam para a mesma posição do array.
Distintas estratégias existem para resolução de colisões; essas provêem 
diferentes garantias de desempenho e frequentemente 
requerem funções hash mais sofisticadas do que aquelas descritas aqui.

Outra categoria de implementações de tabelas hash são os famosos
métodos de \emph{hashing perfeito}.
\index{hashing perfeito}%
Nesses métodos a operação 
 #find(x)# leva $O(1)$ de tempo no pior caso.
 Para conjuntos de dados estáticos isso pode ser realizar ao achar
 \emph{funções hash perfeitas}
\index{funções hash perfeitas}%
\index{função hash!perfeita}%
para os dados; existem funções que mapeiam cada dado de uma única
posição do array. Para dados que mudam com o tempo, métodos 
de hashing perfeito incluem 
 \emph{tabelas hash de dois níveis FKS}
\index{tabela hash de dois níveis}%
\index{tabela hash!dois níveis}%
\cite{fks84,dkkmrt94}
e \emph{hashing cuckoo} \cite{pr04}.
\index{hashing cuckoo }%
\index{tabela hash!cuckoo}%

As funções hash apresentadas neste capítulo estão provavelmente entre 
os métodos práticos atualmente conhecidos que funcionam bem para qualquer
conjunto de dados. Outros métodos bons datam do trabalho pioneiro
de Carter e Wegman que criaram o conceito de \emph{hashing universal}
\index{hashing universal }%
\index{universal!hashing}%
e descreveram várias funções hash para diferentes cenários \cite{cw79}.
Hashing por tabulação, descrita na \secref{tabulation}, foi proposta por Carter
e Wegman \cite{cw79}, mas sua análise, quando aplicada a sondagem linear 
(e vários outros esquemas de tabela hash) 
é creditada a P\v{a}tra\c{s}cu e
Thorup \cite{pt12}.

A ideia de 
 \emph{hashing multiplicativo}
\index{hashing multiplicativo}%
\index{hashing!multiplicativo}%
é muito antiga e parece ser parte do folclore de hashing
 \cite[Section~6.4]{k97v3}.  Porém, a ideia de escolher um 
#z# sendo um número aleatório \emph{ímpar},
e a análise na \secref{multihash} é de Dietzfelbinger \etal\
\cite{dhkp97}.  Essa versão de hashing multiplicativo é uma das mais
simples, mas sua probabilidade de colisão de 
$2/2^{#d#}$ é um fator de 2 maior do que se poderia esperar com uma função 
aleatória de 
$2^{#w#}\to
2^{#d#}$.  O método de \emph{hashing multiplica e soma} 
\index{hashing!multiplica e soma}%
\index{hashing multiplica e soma}%
\[
   h(#x#) = ((#z##x# + b) \bmod 2^{#2w#}) \ddiv 2^{#2w#-#d#}
\]
onde 
#z# e #b# são escolhidos aleatoriamente de $\{0,\ldots,2^{#2w#}-1\}$.
Hashing multiplica e soma tem uma probabilidade de colisão de somente 
$1/2^{#d#}$
\cite{d96}, mas precisa de aritmética de $2#w#$ bits.

Há vários métodos de obter códigos hash a partir de sequências de inteiros com 
#w# bits.  Um método rápido 
\cite{bhkkr99} é a função
\[\begin{array}{l}
  h(#x#_0,\ldots,#x#_{r-1}) \\
   \quad = \left(\sum_{i=0}^{r/2-1} ((#x#_{2i}+#a#_{2i})\bmod 2^{#w#})((#x#_{2i+1}+#a#_{2i+1})\bmod 2^{#w#})\right) \bmod 2^{2#w#}
\end{array}
\]
onde
$r$ é um número par e $#a#_0,\ldots,#a#_{r-1}$ são escolhidos aleatoriamente de 
$\{0,\ldots,2^{#w#}\}$. Isso resulta em um código hash de $2#w#$ bits com probabilidade de colisão 
$1/2^{#w#}$.  Isso pode ser reduzido a um código hash de #w# bits usando 
hashing multiplicativo (ou multiplica e soma). Esse método é rápido porque
requer somente 
 $r/2$ multiplicações de $2#w#$ bits enquanto o método descrito na 
\secref{stringhash} usa $r$ multiplicações.
(As operações $\bmod$ ocorrem implicitamente usando aritmética de #w# e $2#w#$ bits para as adições e multiplicações, respectivamente.)

O método da \secref{polyhash} de usar polinômios em corpos primais para fazer hash de arrays de tamanho variável e strings é atribuído a 
Dietzfelbinger \etal\
\cite{dgmp92}.  Devido ao uso do operador $\bmod$ que usa uma instrução de máquina de alto custo, não é, infelizmente, muito rápida. 
Algumas variantes desse método escolhem o primo
#p# para ser da forma 
$2^{#w#}-1$, que nesse caso o operador $\bmod$ pode ser substituído pelas operações de 
adição (#+#) e AND bit a bit (#&#) \cite[Section~3.6]{k97v2}.
Outra opção é aplicar um dos métodos rápidos para strings de tamanho fixo para blocos de tamanho $c$ para alguma constante $c>1$ e então aplicar o
método de corpo primal à sequência resultante de $\lceil r/c\rceil$ códigos hash.


\begin{exc}
  Uma certa universidade atribui a cada um de seus estudantes números 
  na primeira vez que eles registram-se para qualquer disciplina.
  Esses números são sequências de inteiros que iniciaram em 0 muitos 
  anos atrás e agora estão na casa dos milhões.
  Suponha que temos uma turma de cem alunos do primeiro ano 
  que queremos atribuí-los código hash baseados em seus números de estudantes.
  Faz mais sentido usar os dois primeiros dois dígitos ou os últimos dois
  dígitos do número de estudante deles? Justifique sua resposta.
\end{exc}

\begin{exc}
  Considere o esquema de hashing na \secref{multihash} e suponha 
  $#n#=2^{#d#}$ e $#d#\le #w#/2$.
  \begin{enumerate}
    \item Mostre que, para qualquer escolha de multiplicador #z#, existe #n# valores que têm o mesmo código hash.
      (Dica: isso é fácil e não precisa usar teoria dos números.) 
    \item Dado um multiplicador #z#, descreva #n# valores que têm o mesmo código hash. (Dica: isso é mais difícil e requer um pouco de teoria dos números.) 
  \end{enumerate}
\end{exc}

\begin{exc}
  Prove que o limitante $2/2^{#d#}$ no \lemref{universal-hashing} é o
  melhor limitante possível ao mostrar que, se 
  $x=2^{#w#-#d#-2}$ e 
  $#y#=3#x#$, então $\Pr\{#hash(x)#=#hash(y)#\}=2/2^{#d#}$.  (Dica:
  verifique as representações binárias de 
  $#zx#$ e $#z#3#x#$ e use o fato de que 
  $#z#3#x# = #z#x#+2#z#x#$.)
\end{exc}

\begin{exc}\exclabel{linear-probing}
  Prove o \lemref{linear-probing} usando a versão completa da aproximação de Stirling dada na \secref{factorials}.
\end{exc}

\begin{exc}
  Considere a seguinte versão simplificada do código para adicionar um elemento #x# a uma 
  #LinearHashTable#, que simplesmente guarda #x# na primeira posição 
  do array que esteja #null#.  Explique porque isso poderia ficar muito lento fornecendo um exemplo de uma sequência de 
  operações  $O(#n#)$ #add(x)#, #remove(x)#,
  e #find(x)# que levaria  $O(#n#^2)$ de tempo para executar.
\codeimport{ods/LinearHashTable.addSlow(x)}
\end{exc}

\begin{exc}
  Versões iniciais do método 
   #hashCode()# do Java para a classe #String# não usava todos os caracteres 
   disponíveis em strings longas. Por exemplo, para uma string com dezesseis caracteres, o código hash era computado usando somente oito caracteres com índices pares. Explique porque isso é uma ideia muito ruim fornecendo um exemplo de um grande conjunto de strings que possuem o mesmo código hash.
\end{exc}

\begin{exc}\exclabel{hash-hack-first}
  Suponha que você tem um objeto composto de dois inteiros de #w# bits, #x# e #y#.
  Mostre porque 
  $#x#\oplus#y#$ não é um bom código hash para o seu objeto. 
  Dê um exemplo de um grande conjunto de objetos que teriam código hash 0.
\end{exc}

\begin{exc}
  Suponha que você tem um objeto feito de dois inteiros de #w# bits, #x# e #y#.
  Mostre porque 
   $#x#+#y#$ não seria um bom código hash para o seu objeto.
   Dê um exemplo de grande conjunto de objetos que teriam o mesmo código hash.
  \end{exc}

\begin{exc}\exclabel{hash-hack-last}
  Suponha que você tem um objeto composto de dois inteiros de #w# bits, #x# e #y#.
  Suponha que o código hash para o seu objeto é definido por alguma função hash determinística 
  $h(#x#,#y#)$ que produz um único inteiro de #w# bits. 
  Prove que existe um grande conjunto de objetos que têm o mesmo código hash.
\end{exc}

\begin{exc}
Seja $p=2^{#w#}-1$ para algum inteiro positivo #w#.  Explique porque, para um inteiro positivo $x$
  \[
      (x\bmod 2^{#w#}) + (x\ddiv 2^{#w#}) \equiv x \bmod (2^{#w#}-1) \enspace .
  \]
  \javaonly{
  (Isso resulta em um algoritmo para computar $x \bmod (2^{#w#}-1)$ atribuindo repetidamente 
  \javaonly{\[
    #x = x&((1<<w)-1) + x>>>w#
  \]}
  \cpponly{\[
    #x = x&((1<<w)-1) + x>>w#
  \]}
  até $#x# \le 2^{#w#}-1$.)}
\end{exc}

\begin{exc}
  Ache alguma implementação comumente usada de tabela hash como, por exemplo,
  \javaonly{Java
  Collection Framework #HashMap#,}\cpponly{C++ STL #unordered\_map#,}
  que são implementações da #HashTable# ou #LinearHashTable# deste livro e
  projete um programa que guarda inteiros nessa estrutura de dados
  de forma que haja inteiros #x# tais que #find(x)# gaste tempo linear.
  Isso é, encontre um conjunto de #n# inteiros para os quais existam 
  $c#n#$ elementos cuja hash mapeia para a mesma posição da tabela.

  Dependendo da qualidade da implementação, você pode ser capaz de 
  fazê-los ao inspecionar o código da implementação, ou você pode ter que
  escrever algum código que fazer tentativas de inserções e buscas, medindo
  o tempo que leva para adicionar e achar valores. (Isso pode ser, tem sido, usado para efetuar ataques de negação de serviço em servidores web \cite{cw03}.)
  \index{ataque de complexidade algoritmico}%
\end{exc}
