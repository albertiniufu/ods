\chapter{Skiplists}
\chaplabel{skiplists}

Neste capítulo, discutimos uma elegante estrutura de dados:
a skiplist, que tem uma variedade de aplicações.
Usando uma skiplist podemos implementar uma 
#List# que tem implementações $O(\log n)$ de time para #get(i)#, #set(i,x)#,
#add(i,x)# e #remove(i)#. Podemos também implementar um #SSet# no qual todas as operações rodam em tempo $O(\log #n#)$.
%Finally, a skiplist can be used to implement a #Rope# in which all
%operations run in $O(\log #n#)$ time.

A eficiência de skiplists está o seu uso de 
randomization.
Quando um novo elemento é adicionado a uma skiplist, a skiplists usa um lançamento de uma moeda aleatória para determinar a altura do novo elemento.
O desempenho das skiplists é expresso em termos da esperança do tempo de execução e do comprimento de caminhos percorridos.
Essa esperança (relativo ao valor esperado ou médio) é feita sobre os lançamentos da moeda aleatória usados pela skiplist. Na implementação, os lançamentos são simulados usando um gerador de números (ou bits) pseudo-aleatórios.

\section{A Estrutura Básica}

\index{skiplist}%
Conceitualmente, uma skiplist é uma sequência de listas simplesmente ligadas 
$L_0,\ldots,L_h$. Cada lista $L_r$ contém um subconjunto de itens 
em $L_{r-1}$.  
Iniciamos com a lista de entrada
$L_0$ que contém #n# itens e construímos 
 $L_1$ usando $L_0$, $L_2$ usando $L_1$ e assim por diante.
 Os itens em $L_r$ são obtidos por lançamento de uma moeda para cada elemento #x#
na $L_{r-1}$ e incluindo #x# na $L_r$ se a moeda sai do lado cara.
Esse processo termina quando criamos uma lista $L_r$ que está vazia. 
Um exemplo de uma skiplist é mostrado em \figref{skiplist}.

\begin{figure}
  \begin{center}
    \includegraphics[width=\ScaleIfNeeded]{figs/skiplist}
  \end{center}
  \caption{Uma skiplist contendo sete elementos.}
  \figlabel{skiplist}
\end{figure}

Para um elemento #x# em uma skiplist, chamamos de \emph{altura}
\index{altura!de uma skiplist}%
de #x# o 
maior valor $r$ tal que #x# aparece em $L_r$.  Então, por exemplo
elementos que somente aparecem em $L_0$ tem altura $0$.
Se pararmos alguns momentos para pensar sobre isso, notamos que a altura de #x#
corresponde ao seguinte experimento:
jogar uma moeda repetidamente até que saia coroa. Quantas vezes sai cara?
A resposta, pouco supreendentemente, é que a altura esperada de um nodo é 1.
(Esperamos jogar a moeda duas vezes antes de sair coroa, mas não contamos o último lançamento.) A \emph{altura} de uma skiplist é a altura do seu nodo mais alto.

A cabeça de toda lista é um nodo especial, chamado de 
\emph{sentinela},
\index{nodo sentinela}%
que atua como um nodo dummy para a lista. A propriedade chama de skiplists
é existe um caminho curto, chamado de 
\emph{caminho de busca}, 
\index{caminho de busca!em uma skiplist}%
do sentinela em 
$L_h$ para todo nodo em $L_0$.  Como construir um caminho de busca para um nodo #u# é facil (veja \figref{skiplist-searchpath})
: inicie no canto superior esquerdo da sua skiplist (o sentila em $L_h$)
e sempre vá à direita a menos que ultrapasse #u#, nesse caso você deve descer à lista logo abaixo.

Mais precisamente, para construir o caminho de busca para o nodo #u# em $L_0$,
começamos no sentinela #w# em $L_h$. Depois examinamos #w.next#.
Se 
#w.next# contém um item que aparece antes de #u# em $L_0$, então
fazemos 
 $#w#=#w.next#$.  Caso contrário, seguimos a busca na lista abaixo e buscamos a ocorrência de #w# na lista 
\begin{figure}
  \begin{center}
    \includegraphics[width=\ScaleIfNeeded]{figs/skiplist-searchpath}
  \end{center}
  \caption{O caminho de busca para o nodo contendo $4$ em uma skiplist.}
  \figlabel{skiplist-searchpath}
\end{figure}

O resultado a seguir, que provaremos em 
\secref{skiplist-analysis},
mostra que o caminho de busca é bem curto:

\begin{lem}\lemlabel{skiplist-searchpath}
  O comprimento esperado do caminho de busca para qualquer nodo #u# em 
$L_0$ é no máximo $2\log #n# + O(1) = O(\log #n#)$.
\end{lem}

Um modo eficiente em relação ao espaço para implementar uma 
skiplist é definir um #Node# #u# consistindo de um valor #x#
e um array #next# de ponteiros onde #u.next[i]# aponta para o sucessor de #u#
na lista 
$L_{#i#}$.  Desse jeito, o valor #x# em um nodo é 
\javaonly{referenciado}\cpponly{guardado}\pcodeonly{referenciado} 
somente uma vez #x# pode aparecer em várias listas.

\javaimport{ods/SkiplistSSet.Node<T>}
\cppimport{ods/SkiplistSSet.Node}

As próximas duas seções deste capítulo discutem duas diferentes aplicações de 
skiplists. Em cada uma dessas aplicações, $L_0$ guarda a estrutura principal
(uma lista de elementos or um conjunto ordenado de elementos).
A principal diferença entre essas estruturas está em como um caminho
de busca é navegado; em particular, eles diferem em como eles 
decidem se um caminho de busca ir para a lista abaixo $L_{r-1}$ ou à direita em $L_r$.

\section{#SkiplistSSet#: Um #SSet# Eficiente}
\seclabel{skiplistset}

\index{SkiplistSSet@#SkiplistSSet#}%
Um #SkiplistSSet# usa uma estrutura de skiplist para implementar a interface #SSet#.
Quando usada dessa maneira, a lista 
$L_0$ guarda os elementos de 
#SSet# de modo ordenado. O método #find(x)# funciona seguindo o caminho de busa para o menor valor #y# tal que 
$#y#\ge#x#$:

\codeimport{ods/SkiplistSSet.find(x).findPredNode(x)}

Seguir o caminho de busca para #y# é fácil: quando situado em
algum nodo #u# em $L_{#r#}$ olhamos diretamente para 
 #u.next[r].x#.
 Se
 $#x#>#u.next[r].x#$, então damos um passo à direita em 
$L_{#r#}$; caso contrário, descemos para a lista $L_{#r#-1}$.  
Cada passo (à direita ou embaixo) nessa busca leva apenas um tempo constante,
então, pelo 
\lemref{skiplist-searchpath}, o tempo esperado de execução de #find(x)#
é $O(\log #n#)$.

Antes de podermos adicionar um elemento a uma
 #SkipListSSet#, precisamos de um método para simular o lançamento
 de moedas para determinar a altura, #k#, de um novo nodo. 
Fazemos isso ao escolher um inteiro aleatório, #z#, e contar o número de
$1$s na representação binária de #z#:

\footnote{Esse método não replica exatamente o experimento de lançamento de
moedas pois o valor de #k# será sempre menor que o número de bits em um #int#.
Porém, esse fato terá um impacto negligível a menos que o número de elementos em 
uma estrutura seja muito maior que $2^{32}=4294967296$.}

\codeimport{ods/SkiplistSSet.pickHeight()}

Para implementar o método 
 #add(x)# em uma #SkiplistSSet# procuramos por #x#
 e então dividimos #x# em algumas listas
 $L_0$,\ldots,$L_{#k#}$, onde #k# é selecionado usando o método 
#pickHeight()#. O jeito mais fácil de fazer isso 
é usar um array 
#stack#, que registra os nodos em que o caminho de busca desce de alguma lista
 $L_{#r#}$ em $L_{#r#-1}$.
 Mais precisamente, 
#stack[r]# é o nodo em $L_{#r#}$ onde o caminho de busca desceu em 
$L_{#r#-1}$.  Os nodos que modificamos para inserir  #x# são 
precisamente os nodos 
 $#stack[0]#,\ldots,#stack[k]#$.  O código a seguir implementa esse
 algoritmo para 
#add(x)#:
\label{pg:skiplist-add}
\codeimport{ods/SkiplistSSet.add(x)}

\begin{figure}
  \begin{center}
    \includegraphics[width=\ScaleIfNeeded]{figs/skiplist-add}
  \end{center}
  \caption[Adicionando a uma skiplist]{Adicionando o nodo contendo $3.5$ a uma skiplist.  Os nodos guardados em #stack# estão em destaque.  }
  \figlabel{skiplist-add}
\end{figure}

A remoção de um elemento #x# é feita de modo similar, exceto que
não há necessidade para a 
 #stack# registrar o caminho de busca. A remoção pode ser feita
 conforme vamos seguindo o caminho de busca.
Buscamos por #x# e cada vez que a busca segue para baixo de um novo #u#,
verificamos se 
$#u.next.x#$ é igual a $#x#$ e, caso positivo, tiramos #u# da lista:
\codeimport{ods/SkiplistSSet.remove(x)}

\begin{figure}
  \begin{center}
    \includegraphics[width=\ScaleIfNeeded]{figs/skiplist-remove}
  \end{center}
  \caption{Removing the node containing $3$ from a skiplist.}
  \figlabel{skiplist-remove}
\end{figure}

\subsection{Resumo}

O teorema a seguir resumo o desempenho de skiplists quando usadas para implementar conjuntos ordenados:

\begin{thm}\thmlabel{skiplist}
#SkiplistSSet# implementa a interface #SSet#. Uma #SkiplistSSet# aceita as 
operações #add(x)#, #remove(x)# e #find(x)# em $O(\log #n#)$ de tempo esperado por operação.
\end{thm}

\section{#SkiplistList#: Uma #List# com Acesso Aleatório Eficiente}
\seclabel{skiplistlist}

\index{SkiplistList@#SkiplistList#}%
Uma #SkiplistList# implementa a interface #List# usando uma estrutura skiplist.  Em uma #SkiplistList#, $L_0$ contém os elementos da lista na ordem que eles aparecem na lista. Como em uma 
#SkiplistSSet#, elementos podem ser adicionados, removidos e acessados em  $O(\log
#n#)$ de tempo.

Para isso ser possível, precisamos de um jeito de seguir o caminho de busca para o 
#i#-ésimo em $L_0$.
O jeito mais fácil de fazer isso é definir a noção de \emph{comprimento} de uma aresta em alguma lista, $#L_{#r#}#$.

Definimos o comprimento de toda aresta em 
$L_{0}$ como 1. O comprimento de uma aresta, #e#,
em $L_{#r#}$, $#r#>0$, é definido como a soma dos comprimentos das arestas abaixo de #e# em 
 $L_{#r#-1}$.  De maneira equivalente, o comprimento de #e# é o número de arestas em 
$L_0$ abaixo de #e#.  Veja \figref{skiplist-lengths} para um exemplo
de uma skiplist com os comprimentos de suas arestas mostradas. 
Como as arestas de skiplists são guardadas em arrays, os comprimentos podem ser guardados do mesmo jeito:

\begin{figure}
  \begin{center}
    \includegraphics[width=\ScaleIfNeeded]{figs/skiplist-lengths}
  \end{center}
  \caption{Os comprimentos das arestas em uma skiplist.}
  \figlabel{skiplist-lengths}
\end{figure}

\javaimport{ods/SkiplistList.Node}
\cppimport{ods/SkiplistList.Node}

Essa definição de comprimento tem a útil propriedade de que se estivermos
atualmente em um nodo que está na posição #j#
em $L_0$ e seguimos uma aresta e de comprimento 
$\ell$, então movemos para um nodo cuja posição, em $L_0$,
é $#j#+\ell$.  Desse jeito, enquanto seguirmos um caminho de busca, podemos 
rastrear a posição, #j#, do nodo atual em $L_0$.  
Quando em um nodo 
, #u#, em $L_{#r#}$, iremos à direita se #j# mais o comprimento da aresta 
#u.next[r]# é menor que #i#. Caso contrário, descemos para $L_{#r#-1}$.

\codeimport{ods/SkiplistList.findPred(i)}
\codeimport{ods/SkiplistList.get(i).set(i,x)}

Uma vez que a parte mais difícil das operações #get(i)# e #set(i,x)# é achar 
o 
#i#-ésimo nodo em $L_0$, essas operações rodam em 
$O(\log #n#)$ de tempo.

Adicionar um elemento na 
 #SkiplistList# em uma posição #i# razoavelmente simples. Diferentemente de uma  #SkiplistSSet#, temos certeza que um novo nodo será realmente adicionado, então podemos fazer a adição ao mesmo tempo que procuramos pelo lugar de um novo nodo.
Primeiro pegamos a altura #k# do nodo recentemente inserido #w#
e então seguimos o caminho de busca para #i#.
Toda vez que o caminho de busca desce a partir de 

$L_{#r#}$ com $#r#\le #k#$, inserimos #w#
em $L_{#r#}$.  O último cuidade extra necessário é assegurar que os comprimentos das arestas são atualizadas corretamente. Veja \figref{skiplist-addix}.

\begin{figure}
  \begin{center}
    \includegraphics[width=\ScaleIfNeeded]{figs/skiplist-addix}
  \end{center}
  \caption[Adicionando a uma SkiplistList]{Adicionando um elemento a uma #SkiplistList#.}
  \figlabel{skiplist-addix}
\end{figure}

Note que, cada vez que o caminho de busca desce em um novo #u# em $L_{#r#}$,
o comprimento da aresta 
 #u.next[r]# aumenta em um, pois estamos adicionando um elemento abaixo daquela aresta na posição #i#.

A divisão do nodo #w# entre dois nodos, #u# e #z# funciona como mostrado em  \figref{skiplist-lengths-splice}.  
Ao seguir o caminho de busca já estamos guardando a posição
#j#, de #u# em $L_0$.  Portanto, sabemos que o comprimento da aresta 
de #u# a #w# é $#i#-#j#$.  Também podemos deduzir o comprimento da aresta de #w# a #z# a patir do comprimento $\ell$ de uma aresta de #u# a #z#.
Portanto, podemos dividir em #w# e atualizar os comprimentos das arestas em tempo constante.

\begin{figure}
  \begin{center}
    \includegraphics[scale=0.90909]{figs/skiplist-lengths-splice}
  \end{center}
  \caption[Adição à a SkiplistList]{Atualizar os comprimentos das arestas enquanto dividindo um nodo #w# em uma skiplist.}
  \figlabel{skiplist-lengths-splice}
\end{figure}

Isso parece mais complicado do que é, porque o código é bem simples:

\codeimport{ods/SkiplistList.add(i,x)}
\codeimport{ods/SkiplistList.add(i,w)}

Agora a implementação da operação #remove(i)# em uma #SkiplistList# deve ser óbvia. 
Seguimos o caminho de pesquisa para o nodo na posição #i#. Cada vez que o caminho de busca desce a partir de um nodo #u# no nível #r# nós decrementamos o comprimento
da aresta que deixa #u# naquele nível. Também verificamos se 
 #u.next[r]# é o elemento do #i# e, se o for, o removemos da lista naquele nível. Um exemplo é mostrado em \figref{skiplist-removei}.
\begin{figure}
  \begin{center}
    \includegraphics[width=\ScaleIfNeeded]{figs/skiplist-removei}
  \end{center}
  \caption[Remoção de um elemento de uma SkiplistList]{Remoção de um elemento de uma #SkiplistList#.}
  \figlabel{skiplist-removei}
\end{figure}
\codeimport{ods/SkiplistList.remove(i)}

\subsection{Resumo}

O teorema a seguir resume o desempenho da estrutura de dados 
#SkiplistList#:

\begin{thm}\thmlabel{skiplistlist}
  Uma #SkiplistList# implementa a interface #List#. Uma #SkiplistList#
  aceita a oeprações #get(i)#, #set(i,x)#, #add(i,x)#, e
  #remove(i)# em $O(\log #n#)$ de tempo esperado por operação.
\end{thm}

%\section{Skiplists as Ropes}
%TODO: A section on ropes

\section{Análise de Skiplists}
\seclabel{skiplist-analysis}

Nesta seção, analisaremos a altura esperada, tamanho, e comprimento do caminho
de busca em uma skiplist. Esta seção que conhecimentos básicos de 
probabilidades.
Várias provas são baseadas na seguinte obervação básica sobre lançamentos de 
moedas.

\begin{lem}\lemlabel{coin-tosses}
  \index{coin toss}%
Seja $T$ o número de vezes que uma moeda honesta é lançada, incluindo
a primeira vez que a moeda saiu com a face cara. Então
  $\E[T]=2$.
\end{lem}

\begin{proof}
  Suponha que paramos de lança a moeda na primeira vez que sai cara.
  Defina a variável indicadora
  \[ I_{i} = \left\{\begin{array}{ll}
     0 & \mbox{se a moeda é lançada menos de $i$ vezes} \\
     1 & \mbox{se a moeda é lança $i$ vezes ou mais} 
     \end{array}\right.
  \]
  Note que 
   $I_i=1$ se, e somente se, os primeiros $i-1$ lançamentos saem coroa,
  então $\E[I_i]=\Pr\{I_i=1\}=1/2^{i-1}$.  Observe que $T$, o número
  total de lançamentos, pode ser escrito como 
   $T=\sum_{i=1}^{\infty} I_i$.
  Portanto,
  \begin{align*}
    \E[T] & =  \E\left[\sum_{i=1}^\infty I_i\right] \\
     & =  \sum_{i=1}^\infty \E\left[I_i\right] \\
     & =  \sum_{i=1}^\infty 1/2^{i-1} \\
     & =  1 + 1/2 + 1/4 + 1/8 + \cdots \\
     & =  2 \enspace .   \qedhere
  \end{align*} 
\end{proof}

Os próximos dois lemas afirmam que skiplists tem tamanho linear:

\begin{lem}\lemlabel{skiplist-size1}
  O número esperado de nodos em um skiplist contendo 
   $#n#$ elementos,
   sem incluir as ocorrências da sentinela, é 
  $2#n#$.
\end{lem}

\begin{proof}
  A probabilidade que qualquer elemento em particular, #x#, seja incluído
  na lista 
  $L_{#r#}$ é $1/2^{#r#}$, então o número esperado de nodos em $L_{#r#}$
  é $#n#/2^{#r#}$.\footnote{Veja \secref{randomization} para ver como isso é obtido usando variáveis indicadores e linearidade de esperança.}
  Portanto, o número esperado total de nodos em todas as listas é 
  \[ \sum_{#r#=0}^\infty #n#/2^{#r#} = #n#(1+1/2+1/4+1/8+\cdots) = 2#n# \enspace . \qedhere \]
\end{proof}

\begin{lem}\lemlabel{skiplist-height}
  A altura esperada de uma skiplist contendo #n# elementos é até 
  $\log #n# + 2$.
\end{lem}

\begin{proof}
  Para cada $#r#\in\{1,2,3,\ldots,\infty\}$, 
  defina a variável indicadora aleatória
  \[ I_{#r#} = \left\{\begin{array}{ll}
     0 & \mbox{se $L_{#r#}$ está vazia } \\
     1 & \mbox{se $L_{#r#}$ não está vazia}
     \end{array}\right.
  \]
  A altura, #h# da skiplist é então dada por
  \[
       #h# = \sum_{r=1}^\infty I_{#r#} \enspace .
  \]
  Note que $I_{#r#}$ nunca é maior que o comprimento, $|L_{#r#}|$, de $L_{#r#}$, então
  \[
     \E[I_{#r#}] \le \E[|L_{#r#}|] = #n#/2^{#r#} \enspace .
  \]
  Portanto, temos
  \begin{align*}
       \E[#h#] &= \E\left[\sum_{r=1}^\infty I_{#r#}\right] \\
        &= \sum_{#r#=1}^{\infty} E[I_{#r#}] \\
        &= \sum_{#r#=1}^{\lfloor\log #n#\rfloor} E[I_{#r#}]
                 + \sum_{r=\lfloor\log #n#\rfloor+1}^{\infty} E[I_{#r#}]  \\
        &\le \sum_{#r#=1}^{\lfloor\log #n#\rfloor} 1
                 + \sum_{r=\lfloor\log #n#\rfloor+1}^{\infty} #n#/2^{#r#} \\
        &\le \log #n#
                 + \sum_{#r#=0}^\infty 1/2^{#r#} \\
        &= \log #n# + 2 \enspace . \qedhere
  \end{align*}
\end{proof}

\begin{lem}\lemlabel{skiplist-size2}
  O número esperado de nodos em uma skiplist contendo 
   $#n#$ elementos, incluindo todas as ocorrências da sentinela é 
  $2#n#+O(\log #n#)$.
\end{lem}

\begin{proof}
  Segundo \lemref{skiplist-size1}, o número esperado de nodos, sem incluir o sentinela, 
  é $2#n#$.  O número de ocorrências do sentinela é igual à altura 
  $#h#$ da skiplist então, segundo
  \lemref{skiplist-height}, o número esperado de ocorrências da sentinela é até 
  $\log #n#+2 = O(\log #n#)$.
\end{proof}

\begin{lem}
O comprimento esperado de um caminho de busca em uma skiplist é até
$2\log #n# + O(1)$.
\end{lem}

\begin{proof}
  O jeito mais fácil de ver issso é considerar o 
   \emph{caminho de busca reverso}. Esse caminho inicia-se no predecessor de #x#
   em 
   $L_0$.  Em qualquer momento, se o caminho pode subir um nível, então ele o faz.
   Se ele não puder subir um nível, então vai à esquerda. Considerando isso 
   por alguns momentos vemos que o caminho de busca reverso para #x# é idêntico
   ao caminho de busca para #x#, exceto que é invertido.

   O número de nodos que o caminho de busca reverso visita em um dado
   nível #r# é relacionado ao seguinte experimento: lance uma moeda.
   Se a moeda cai cara, então suba um nível e pare. Caro contrário,
   vá à esquerda e repita o experimento. O número de lançamentos
   obtido dessa forma representa o número de passos à esquerda que um 
   caminho de busca invertido faz em um dado nível.
  \footnote{Note que isso pode superestimar o número de passos à
  esquerda pois o experimento deve terminar ou em um lançamento cara ou
  quando o caminho de busca chega no sentinela, aquele que vier primeiro.
  Isso não é um problema porque o lema está somente apresentando um limitante
  superior.}
   \lemref{coin-tosses} afirma que o número esperado de lançamento de moedas
   antes da primeira cada é 1. 
Seja 
   $S_{#r#}$ o número de passos que uma caminho de busca (direta) usa no
   nível
  $#r#$ para ir à direita.  Acabamos de mostrar que $\E[S_{#r#}]\le
  1$.  Além disso, $S_{#r#}\le |L_{#r#}|$, pois não podemos fazer mais passos em
  $L_{#r#}$ que o próprio comprimento de $L_{#r#}$, então
  \[
    \E[S_{#r#}] \le \E[|L_{#r#}|] = #n#/2^{#r#} \enspace .
  \]
  Podemos agora terminar como na prova de 
  \lemref{skiplist-height}.
  Seja $S$ o comprimento do caminho de busca para algum nodo #u# em uma skiplist e seja $#h#$ a altura da skiplist. Então
  \begin{align*}
      \E[S] 
         &= \E\left[ #h# + \sum_{#r#=0}^\infty S_{#r#} \right] \\
         &= \E[#h#] + \sum_{#r#=0}^\infty \E[S_{#r#}]  \\
         &= \E[#h#] + \sum_{#r#=0}^{\lfloor\log #n#\rfloor} \E[S_{#r#}] 
              + \sum_{#r#=\lfloor\log #n#\rfloor+1}^\infty \E[S_{#r#}] \\
         &\le \E[#h#] + \sum_{#r#=0}^{\lfloor\log #n#\rfloor} 1
              + \sum_{r=\lfloor\log #n#\rfloor+1}^\infty #n#/2^{#r#} \\
         &\le \E[#h#] + \sum_{#r#=0}^{\lfloor\log #n#\rfloor} 1
              + \sum_{#r#=0}^{\infty} 1/2^{#r#} \\
         &\le \E[#h#] + \sum_{#r#=0}^{\lfloor\log #n#\rfloor} 1
              + \sum_{#r#=0}^{\infty} 1/2^{#r#} \\
         &\le \E[#h#] + \log #n# + 3 \\
         &\le 2\log #n# + 5  \enspace . \qedhere
  \end{align*}
\end{proof}

O teorema a seguir resume os resultados nessa seção:
\begin{thm}
  Uma skiplist contendo
$#n#$ elementos tem tamanho esperado $O(#n#)$ e o comprimento esperado
  do caminho de busca para qualquer elemento é até
$2\log #n# + O(1)$.
\end{thm}

%\section{Iteration and Finger Search}

%TODO: Write this section

\section{Discussão e Exercícios}

Skiplists foram inicialmente propostas por \cite{p91} que também apresentou várias aplicações e extensões de skiplists 
\cite{p89}.  Desde então ela tem sido extensivamente estudada.
Vários pesquisadores tem realizado análises bem precisas do comprimento
esperado e da variância do comprimento do caminho de busca para o
#i#-ésimo elemento em uma skiplist \cite{kp94,kmp95,pmp92}.
Versões determinísticas
 \cite{mps92}, versões tendenciosas \cite{bbg02,esss01},
e versões auto-ajustáveis \cite{bdl08} de skiplists têm surgido.  Implementações de skiplists têm sido escritas para várias linguagens e frameworks e usadas em sistemas de bancos de dados open-source \cite{skipdb,redis}. Uma variante de skiplists é usada nas estruturas do gerenciador de processos do kernel do sistema operacional HP-UDX A variant of skiplists is used in the HP-UX
\cite{hpux}.
\javaonly{Skiplists são até parte da Java 1.6 API \cite{oracle_jdk6}.}


\begin{exc}
  Desenhe os caminhos de busca para 2.5 e 5.5 na skiplists em 
  \figref{skiplist}.
\end{exc}

\begin{exc}
  Desenhe a adição de valores 0.5 (com altura de 1) e então 3.5 (com
  uma altura de 2) para a skiplist em 
  \figref{skiplist}.
\end{exc}

\begin{exc}
  Desenhe a remoção de valores 1 e então 3 da skiplist em 
   \figref{skiplist}.
\end{exc}

\begin{exc}
  Desenhe a execução de 
   #remove(2)# na #SkiplistList# em 
  \figref{skiplist-lengths}.
\end{exc}

\begin{exc}
  Desenhe a execução de 
   #add(3,x)# na #SkiplistList#
  da \figref{skiplist-lengths}.  Assuma que #pickHeight()# seleciona uma altura
  de 4 para o nodo recentemente criado. 
\end{exc}

\begin{exc}\exclabel{skiplist-changes}
  Mostre que, durante uma operação
  #add(x)# ou #remove(x)#, o número esperado de ponteiros em 
  uma #SkiplistSet# que é alterado é constante. 
\end{exc}

\begin{exc}\exclabel{skiplist-opt}
  Suponha que, em vez de promover um elemento de 
   $L_{i-1}$ em $L_i$ usando um lançamento de moeda
  , fizermos a promoção com uma probabilidade $p$, $0 <
  p < 1$.
  \begin{enumerate}
   \item Mostre que, com essa modificação, o comprimento esperado de um caminho de busca é até $(1/p)\log_{1/p} #n# + O(1)$.
   \item Qual é o valor esperado de $p$ que minimiza a expressão precedente? 
   \item Qual é altura esperada da skiplist? 
   \item Qual é o número esperado de nodos na skiplist? 
  \end{enumerate}
\end{exc}

\begin{exc}\exclabel{skiplist-opt-2}
O método #find(x)# em uma a #SkiplistSet# às vezes faz 
  \emph{comparações redundantes}; essas ocorrem quando #x# é comparado ao mesmo valor mais de uma vez. Eles podem ocorrer quando, para algum nodo #u# 
  $#u.next[r]# = #u.next[r-1]#$.  Mostre quando essas comparações redundantes acontecem e modifique 
  #find(x)# para que sejam evitadas. Analise o número esperado de comparações feito pelo seu método modificado #find(x)#. 
\end{exc}

\begin{exc}
  Projete e implemente uma versão de uma skiplist que implemente a interface
  #SSet#, mas também permite acesso rápido a eles pelo rank.
  Isto é, também aceita a função 
  #get(i)#, que retorna o elemento cujo rank é #i#
   #i# em $O(\log #n#)$ de tempo esperado. (O rank
   de um elemento #x# em uma #SSet# é o número de elementos em #SSet#
   que são menores que #x#.)
\end{exc}

\begin{exc}
  \index{finger}%
  \index{finger search!in a skiplist}%
  Um \emph{finger} em uma skiplist é um array que guarda a sequência de nodos em um caminho de busca no qual o caminho de busca desce.
  (A variável #stack# no código #add(x)# na página~\pageref{pg:skiplist-add} é um finger;  os nodos com sombra em 
  \figref{skiplist-add} mostra o conteúdo do finger.) Pode-se pensar de um finger como estar apontando o caminho para um nodo na lista mais baixa, $L_0$. 

  Uma \emph{busca finger} implementa a operação #find(x)# usando um 
  finger, ao percorrer a lista usando finger até chegar em um nodo 
  #u# tal que $#u.x# < #x#$ e $#u.next#=#null#$ ou $#u.next.x# >
  #x#$ e então fazer uma busca normal por #x# començando de #u#. 
  É possível provar que o número esperado de passos necessários
  para uma busca finger é 
   $O(1+\log r)$, onde $r$ é o número de valores em 
  $L_0$ entre #x# e o valor apontado pelo finger.

  Implemente uma subclasse de #Skiplist# chamada #SkiplistWithFinger# que 
  implementa operações #find(x)# usando um finger interno. Essa subclasse
  guarda um finger, que é então usado para que toda operação 
  #find(x)# seja implementada como uma busca finger. 
  Durante cada operação 
  #find(x)# o finger é atualizado tal que cada operação  
  #find(x)# usa como ponto de partida 
  , um finger que aponta para o resultado da operação 
  #find(x)# prévia.
\end{exc}

\begin{exc}\exclabel{skiplist-truncate}
  Escreva um método
  #truncate(i)# que trunca uma #SkiplistList#
  na posição #i#.  Após a execução desse método, o tamanho da lista é #i#
  e contém somente os elementos nos índices
  $0,\ldots,#i#-1$.  O valor retornado é outra #SkiplistList# que 
  contém os elementos nos índices 
   $#i#,\ldots,#n#-1$.  Esse método deve rodar em 
   $O(\log #n#)$ de tempo.
\end{exc}

\begin{exc}
  Escreva um método para a
  #SkiplistList# chamado #absorb(l2)#, que recebe como argumento uma 
  #SkiplistList#, #l2#, a vazia e coloca seu conteúdo na estrutura receptora.  
Por exemplo, se 
  #l1# contém $a,b,c$
  e #l2# contém $d,e,f$, então após chamar #l1.absorb(l2)#, #l1#
  conterá $a,b,c,d,e,f$ e #l2# estará vazia. Esse método deve rodar em 
  $O(\log #n#)$ de tempo.
\end{exc}

\begin{exc}
  Usando as ideias da lista eficiente em espaço, #SEList#, projete
  e implemente uma #SSet# eficiente em espaço #SSet#s, #SESSet#.
  Para isso, guarde os dados, em ordem, em uma #SEList# e armazene os blocos dessa
  #SEList# em uma #SSet#. Se a implementação original #SSet# usa
  $O(#n#)$
  de espaço para guardar #n# elementos, então a #SESSet# irá usar espaço suficiente 
  para #n# elementos mais 
  $O(#n#/#b#+#b#)$ de espaço desperdiçado.
\end{exc}

\begin{exc}
  Usando uma #SSet# com estrutura básica, projete e implemente uma
  aplicação que lê um arquivo de texto (potencialmente enorme) e permite
  você buscar, iterativamente, por qualquer substring contida no texto.
  Conforme o usuário digita a consulta, uma parte do texto que corresponde
  à busca (se houver) deve aparecer como resultado. 

  \noindent  Dica 1: toda substring é um prefixo de algum sufixo, então basta guarda todos os sufixos do arquivo de texto. 

  \noindent Dica 2: qualquer sufixo pode ser representado como um único inteiro indicando onde o sufixo começa no texto.

  \noindent Teste sua aplicação em textos maiores, tais como alguns livros disponibilizados pelo 
  Projeto Gutenberg \cite{gutenberg}.  Se feito corretamente, 
  a sua aplicação será bem rápida; não deve have atraso considerável
  entre digitar letras e ver os resultados.
\end{exc}

\begin{exc}
  \index{skiplist!versus árvore binária de busca}%
  \index{árvore binária de busca!versus skiplist}%
  (Este exercício deve ser feito após a leitura sobre árvores binárias de busca
  em
   \secref{binarysearchtree}.)  Compare skiplists com árvores binárias de busca
  das seguintes maneiras:
  \begin{enumerate}
     \item Explique como algumas arestas de uma skiplist leva a uma estrutura que parece uma árvore binária e é similar a uma árvore de busca binária. 
     \item Skiplists e árvores de busca binária usam aproximadamente o mesmo número de ponteiros (2 por nodo). Contundo, skiplists fazer melhor uso desses ponteiros. Explique porque.
  \end{enumerate}
\end{exc}
