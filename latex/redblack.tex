%%merge = união
%%split = repartição
\chapter{Árvores Rubro-Negras}
\chaplabel{redblack}

\index{binary search tree!red-black}%
\index{árvore binária de busca!rubro-negra}%
\index{red-black tree}%
\index{árvore rubro-negra}%
Neste capítulo, apresentamos as árvores rubro-negras (em inglês, red-black trees), 
uma versão de árvores binárias de busca com altura logarítmica.
Árvores rubro-negras são as estruturas de dados mais amplamente usadas.
Elas aparecerem como estrutura de busca primária em muitas implementações, incluindo a 
Java Collections Framework
e várias implementações da 
C++ Standard Template Library. Elas também são usadas dentro do kernel Linux. 
Existem várias razões para a popularidade das árvores rubro-negras:
\begin{enumerate}
\item Uma árvore rubro-negra com #n# valores tem altura no máximo $2\log #n#$.
\item As operações #add(x)# e #remove(x)# em uma árvore rubro-negra rodam em tempo 
    $O(\log #n#)$ no \emph{pior caso}.
\item O número amortizado de rotações realizado durante uma operação #add(x)#
   ou #remove(x)# é constante.
\end{enumerate}
As duas primeiras propriedades põem árvores rubro-negras a frente de skiplists,
treaps e árvores scapegoat.
Skiplists e treaps dependem de randomização e seus tempos de execução $O(\log #n#)$
somente são esperados. Árvores scapegoat têm um limitante garantido na altura, mas #add(x)# e #remove(x)# rodam em tempo amortizado $O(\log #n#)$. 
A terceira propriedade é apenas a cereja do bolo. Ela nos diz que o tempo necessário para adicionar ou remover um elemento #x# é minúsculo em relação ao tempo que leva para achar #x#. 
\footnote{Note que skiplists e
treaps também têm essas propriedades de forma esperada. Veja os
exercícios~\ref{exc:skiplist-changes} e \ref{exc:treap-rotates}.}

Entretanto, as boas propriedades de árvores rubro-negras têm um preço: complexidade de implementação. Manter um limitante de $2\log #n#$ na altura não é fácil. 
Isso exige uma análise cuidadosa de vários casos.
Precisamos assegurar que a implementação siga exatamente os passos corretos em cada caso.
Uma rotação mal posicionada ou um alteração de cor errada causa um bug que pode ser muito difícil de entender e resolver.

Em vez de pular diretamente à implementação da árvore rubro-negra,
primeiro construímos uma base por meio de uma estrutura de dados diretamente relacionada: árvore 2-4. Isso fornecerá entendimento de como árvores rubro-negras
foram descobertas e porque é possível manter essa estrutura eficientemente.

\section{Árvore 2-4}
\seclabel{twofour}

Uma árvore 2-4 é uma árvore enraizada com as seguintes propriedades: 
\begin{prp}[height]
  Todas as folhas têm a mesma profundidade. 
\end{prp}
\begin{prp}[degree]
  Todo nodo interno tem 2,3 ou 4 filhos.
\end{prp}
Um exemplo de uma árvore 2-4 é mostrado na \figref{twofour-example}.
\begin{figure}
  \begin{center}
    \includegraphics[scale=0.90909]{figs/24rb-2}
  \end{center}
  \caption{Uma árvore 2-4 de altura 3.}
  \figlabel{twofour-example}
\end{figure}
As propriedades de uma árvore 2-4 implicam que sua altura é logarítmica no número de folhas:
\begin{lem}\lemlabel{twofour-height}
  A árvore 2-4 com #n# folhas tem altura até $\log #n#$.
\end{lem}

\begin{proof}
  O limitante inferior de 2 no número de filhos de um nodo interno
  implica que, se altura de uma árvore 2-4 for $h$, então ela tem pelo menos 
  $2^h$ folhas.  Em outras palavras,
  \[
     #n# \ge 2^h \enspace .
  \]
  Aplicando logaritmos em ambos lados dessa desigualdade resulta em 
  $h \le \log #n#$.
\end{proof}

\subsection{Adição de uma Folha}

Adicionar uma folha a uma 
árvore 2-4 é fácil (veja a \figref{twofour-add}).  Se
queremos adicionar uma folha
#u# coma filha de um nodo #w# no penúltimo nível,
então simplesmente fazemos #u# um filho de #w#. Isso certamente
mantém a 
propriedade da altura, mas pode violar a propriedade do grau; se #w#
tinha quatro filhos antes de adicionar #u#, então #w# agora tem cinco filhos.
Neste caso, \emph{repartimos}
\index{repartição}%
#w# em dois nodos, #w# e #w#', com dois e três filhos, respectivamente. 
Mas agora #w#' não tem pai, então recursivamente tornamos 
#w#' um filho do pai de #w#.  Novamente, isso pode fazer com que o pai de #w#
tenha muitos filhos e, dessa forma, teremos que reparti-los.
Esse processo segue repetidamente até alcançar um nodo que tem menos que quatro filhos, ou até repartimos a raiz #w# em dois nodos #r# e #r'#. No último caso,
fazemos uma nova raiz que tem #r# e #r'# como filhos.
Isso simultaneamente aumenta a profundidade de todas as folhas e 
assim mantém propriedade da altura.

\begin{figure}
  \begin{center}
   \begin{tabular}{c}
     \includegraphics[scale=0.90909]{figs/24tree-add-1} \\
     \includegraphics[scale=0.90909]{figs/24tree-add-2} \\
     \includegraphics[scale=0.90909]{figs/24tree-add-3}
   \end{tabular}
  \end{center}
  \caption[Adição de uma folha em uma árvore 2-4]{Adição de uma folha a uma árvore 2-4. Esse processo para após um repartição porque #w.parent# tem um grau menor que 4 antes da adição.}
  \figlabel{twofour-add}
\end{figure}

Como a altura da árvore 2-4 nunca é maior que $\log #n#$, o 
processo de adicionar uma folha termina após no máximo
$\log #n#$ passos.

\subsection{Remoção de uma Folha}

Remoção de uma folha de uma 
árvore 2-4 é um pouco mais complicada (veja
a \figref{twofour-remove}).  Para remover uma folha #u# de seu pai #w#, 
então simplesmente a removemos. 
Se #w# tivesse somente dois filhos antes à remoção de #u#,
então #w# é deixado com somente um filho, o que viola a propriedade de grau.

\begin{figure}
  \begin{center}
   \begin{tabular}{c}
     \includegraphics[height=\FifthHeightScaleIfNeeded]{figs/24tree-remove-1} \\
     \includegraphics[height=\FifthHeightScaleIfNeeded]{figs/24tree-remove-2} \\
     \includegraphics[height=\FifthHeightScaleIfNeeded]{figs/24tree-remove-3} \\
     \includegraphics[height=\FifthHeightScaleIfNeeded]{figs/24tree-remove-4} \\
     \includegraphics[height=\FifthHeightScaleIfNeeded]{figs/24tree-remove-5} \\
   \end{tabular}
  \end{center}
  \caption[Remoção de uma folha de uma árvore 2-4]{Remoção de uma folha de uma 
    árvore 2-4.  Esse processo segue até a raiz pois cada 
    ancestral de #u# e seus irmãos têm somente dois filhos.} 
  \figlabel{twofour-remove}
\end{figure}

Para corrigir isso, olhamos em um irmão de #w#, o #w'#. O nodo #w'#
com certeza existe pois o pai de #w# tem pelo menos dois filhos. Se #w'#
tem três ou quatro filhos, então pegamos um desses filhos de #w'#
e o transferimos para #w#. Agora #w# tem dois filhos e #w'# tem dois ou
três filhos e terminamos o processo.

Por outro lado, se #w'# tem somente dois filhos, então fazemos uma fusão (em inglês, \emph{merge})
\index{fusão}%
de #w# e #w'# em um único nodo, #w#, que tem três filhos. A seguir, 
recursivamente removemos #w'# do pai de #w'#.  Esse processo
termina guando alcançamos um nodo #u#, onde #u# ou um irmão seu tem mais de
dois filhos, ou quando chegamos à raiz. No último caso, se a raiz 
é deixada com somente um filho, então removemos a raiz e fazemos seu filho a
nova raiz.

Novamente, isso simultaneamente reduz a altura de todas as folhas e portanto
mantém a propriedade de altura.

Como a altura da árvore não passa de $\log #n#$, o processo de remover uma folha acaba após no máximo $\log #n#$ passos.

\section{#RedBlackTree#: Uma Árvore 2-4 Simulada}
\seclabel{redblacktree}

Uma árvore rubro-negra é uma árvore binária de busca em que cada nodo #u#
tem uma \emph{cor}
\index{cor}%
que é 
\emph{vermelha} ou \emph{preta}.  Vermelho é representado pelo valor $0$
e preto pelo valor $1$. 
\index{nodo vermelho}%
\index{nodo preto}%
\javaimport{ods/RedBlackTree.red.black.Node<T>}
\cppimport{ods/RedBlackTree.RedBlackNode.red.black}

Antes e depois de qualquer operação em uma árvore rubro-negra, as 
seguintes duas propriedades são satisfeitas. Cada propriedade é definida em
termos das cores vermelha e preta e em termos dos valores numéricos 0 e 1.

\begin{prp}[black-height]
  \index{propriedade da altura preta}%
  Há o mesmo número de nodos pretos em todo caminho da raiz para a folha.
  (A soma das cores em qualquer caminho da raiz a uma folha é a mesma.)
\end{prp}

\begin{prp}[no-red-edge]
  \index{propriedade de nenhuma aresta vermelha}%
  Não há nodos vermelhos adjacentes. (Para qualquer nodo #u#, exceto a raiz,
  $#u.colour# + #u.parent.colour# \ge 1$.)
\end{prp}
Note que sempre podemos colorir a raiz #r# de uma árvore rubro-negra de preto 
sem violar nenhuma dessas duas propriedades, então iremos
assumir que a raiz é preta e os algoritmos para atualizar uma árvore rubro-negra
irão manter isso.
Outro truque que simplifica na implementação
é tratar os nodos externos (representados por #nil#) como nodos pretos.
Dessa forma, todo nodo real, #u#, de uma árvore rubro-negra tem exatamente
dois filhos, cada qual com uma cor bem definida. Um exemplo de uma
árvore rubro-negra é mostrado na \figref{redblack-example}.

\begin{figure}
  \begin{center}
    \includegraphics[scale=0.90909]{figs/24rb-1}
  \end{center}
  \caption[Uma árvore rubro-negra]{Um exemplo de uma árvore rubro-negra com uma altura preta de 3. Nodos externos (#nil#) são desenhados como quadrados.}
  \figlabel{redblack-example}
\end{figure}


\subsection{Árvores Rubro-Negras e Árvores 2-4}

À primeira vista, pode parecer surpreendente que uma árvore rubro-negra possa
ser atualizada eficientemente para manter as propriedades de altura preta e de 
nenhuma aresta vermelha e parece estranho considerar essas propriedades como úteis.
Entretanto, árvores rubro-negras foram projetadas para serem uma simulação 
eficiente das árvores 2-4 na forma de árvores binárias.

Veja a \figref{twofour-redblack}.
Considere qualquer árvore rubro-negra, $T$, com #n# nodos e realize a seguinte 
transformação: remova cada nodo vermelho #u# e conecte os dois filhos 
de #u# diretamente ao pai (preto) de #u#.
Após essa transformação, a árvore resultante $T'$ possui somente nodos pretos.
\begin{figure}
  \begin{center}
    \begin{tabular}{cc}
      \includegraphics[scale=0.90909]{figs/24rb-3} \\
      \includegraphics[scale=0.90909]{figs/24rb-2}
    \end{tabular}
  \end{center}
  \caption{Toda árvore rubro-negra tem uma árvore 2-4 correspondente.}
  \figlabel{twofour-redblack}
\end{figure}

Todo nodo interno em $T'$ tem dois, três ou quatro filhos: um nodo preto 
que tinha dois filhos pretos continuará com
dois filhos pretos após essa transformação.
Um nodo preto que tinha com um nodo preto e um vermelho terá três
filhos após essa transformação.
Um nodo preto que começou com dois filhos vermelhos terá quatro filhos após
essa transformação. Além disso, a propriedade de altura preta garante
que todo caminho da raiz até a folha em $T'$ tem o mesmo comprimento.
Em outras palavras, $T'$ é uma árvore 2-4!

A árvore 2-4 $T'$ tem $#n#+1$ folhas que correspondem
a $#n#+1$ nodos externos da árvore rubro-negra. Portanto, essa árvore
tem altura de até 
$\log (#n#+1)$. Agora, todo caminho da raiz para uma folha na árvore 2-4 corresponde
a um caminho da raiz da árvore rubro-negra $T$ a um nodo externo.

O primeiro e último nodo nesse caminho são pretos e no máximo um de cada dois
nodos internos é vermelho, então esse caminho tem no máximo
$\log(#n#+1)$ nodos pretos e no máximo 
$\log(#n#+1)-1$ nodos vermelhos. Portanto, o caminho mais longo da raiz para qualquer nodo \emph{interno} em $T$ é no máximo 
\[
   2\log(#n#+1) -2 \le 2\log #n# \enspace ,
\]
para todo
$#n#\ge 1$.  Isso é a propriedade mais importante das 
árvores rubro-negras:
\begin{lem}
A altura de uma árvore rubro-negra com #n# nodos é no máximo $2\log #n#$.
\end{lem}

Agora que estudamos a relação entre as árvores 2-4 e as 
árvores rubro-negras, não é difícil de acreditar que podemos manter eficientemente uma árvore rubro-negra ao adicionar e remover elementos. 

Vimos que a adição de elementos em uma  #BinarySearchTree#
 pode ser feita pela adição de uma nova folha. Portanto, para implementar 
#add(x)# em uma árvore rubro-negra precisamos de um método de simular
a repartição de um nodo com cinco filhos em uma 
árvore 2-4.  O nodo da árvore 2-4 com cinco filhos é representado por
um nodo preto que tem dois filhos vermelhos, um dos quais também tem um filho
vermelho. Podemos ``reparticionar'' esse nodo pintando ele de vermelho
e pintando seus dois filhos de preto. 
Um exemplo disso é mostrado na \figref{rb-split}.

\begin{figure}
  \begin{center}
   \begin{tabular}{c}
     \includegraphics[scale=0.90909]{figs/rb-split-1} \\
     \includegraphics[scale=0.90909]{figs/rb-split-2} \\
     \includegraphics[scale=0.90909]{figs/rb-split-3} \\
   \end{tabular}
  \end{center}
  \caption[Simulando uma árvore 2-4]{Simulando uma operação de reparticionamento de uma árvore 2-4 durante a adição em uma árvore rubro-negra. (Isso simula
  a adição da árvore 2-4 mostrada na \figref{twofour-add}.)}
  \figlabel{rb-split}
\end{figure}

De modo similar, implementar 
#remove(x)# exige um método de unir dois nodos e emprestar um filho de um irmão.
A união de dois nodos é o inverso da repartição (mostrado na 
\figref{rb-split}), e envolve pintar dois irmãos vermelhos de 
preto e pintar o pai (que é vermelho) de preto. Emprestar um 
irmão é o procedimento mais complicado
e envolve ambas as rotações e pintar os nodos.

Obviamente, durante tudo isso devemos ainda manter a propriedade de nenhuma aresta 
vermelha e a propriedade da altura preta. Enquanto não é mais surpreendente 
que isso pode ser feito, existe um grande número de casos que devem ser considerados
se tentarmos fazer uma simulação direta de uma 
árvore 2-4 com uma árvore rubro-negra. 
Eventualmente, torna-se mais simples desconsiderar a árvore 2-4 e trabalhar
diretamente com a manutenção das propriedades da árvore rubro-negra.

\subsection{Árvores Rubro-Negras Pendentes à Esquerda}

\index{árvore rubro-negra}%
\index{árvore rubro-negra pendente à esquerda}%
Não existe uma única definição de 
árvores rubro-negras. Em vez disso, existe uma família
de estruturas que conseguem manter as propriedades da altura preta
e nenhuma aresta vermelha 
durante as operações #add(x)# e #remove(x)#. Estruturas diferentes fazem isso diferentemente. 
Aqui, implementamos uma estrutura de dados que chamamos de 
#RedBlackTree#.
\index{RedBlackTree@#RedBlackTree#}%
Essa estrutura implementa uma variante de árvore rubro-negra que satisfaz
uma propriedade adicional:
\begin{prp}[left-leaning]\prplabel{left-leaning}\prplabel{redblack-last}
  \index{propriedade de pender à esquerda}%
  Em qualquer nodo #u#, se #u.left# for preto, então #u.right# é preto. 
\end{prp}
Note que a árvore rubro-negra mostrada na 
\figref{redblack-example}  não satisfaz a propriedade de pender 
à esquerda; 
ela é violada pelo pai do nodo vermelho no caminho mais à esquerda.

A razão para manter a propriedade de pender à esquerda é que ela reduz
o número de casos encontrados durante a atualização da árvore nas operações
#add(x)#
e #remove(x)#.  Em termos de uma árvore 2-4, isso implica que toda árvore 2-4 tem uma representação única: um nodo de grau dois se torna um nodo preto com dois filhos pretos.
Um nodo de grau três se torna um nodo preto cujo filho à esquerda é vermelho 
e cujo filho à direita é preto.
Um nodo de grau quatro se torna um nodo preto com dois filhos vermelhos.

Antes de descrevermos a implementação de 
 #add(x)# e #remove(x)# em detalhes, primeiro apresentamos alguma sub-rotinas
 simples usadas por esses métodos que estão ilustradas na 
 \figref{redblack-flippullpush}.  As duas primeiras sub-rotinas são para manipular
 as cores e presentar a propriedade da altura preta.
 O método #pushBlack(u)# recebe como entrada um nodo preto #u#
que tem dois filhos vermelhos e pinta #u# de vermelho e seus dois filhos de preto.
O método #pullBlack(u)# inverte essa operação
\codeimport{ods/RedBlackTree.pushBlack(u).pullBlack(u)}

\begin{figure}
  \begin{center}
    \includegraphics[width=\ScaleIfNeeded]{figs/flippullpush}
  \end{center}
  \caption{Operações de \emph{flips}, \emph{pulls} e \emph{pushes}}
  \figlabel{redblack-flippullpush}
\end{figure}

O método
#flipLeft(u)# troca as cores de #u# e #u.right#
e então realiza uma rotação à esquerda em #u#.
Esse método inverte as cores desses dois nodos assim como seu relacionamento
pai-filho:
\codeimport{ods/RedBlackTree.flipLeft(u)}
A operação #flipLeft(u)# é especialmente útil ao restaurar a
propriedade de pender à esquerda em um nodo #u# que a viola (porque 
 #u.left# é preto e #u.right# é vermelho).
 Nesse caso especial, temos a certeza de que essa operação preserva 
 as propriedades de altura preta e de nenhuma aresta vermelha.
 A operação
#flipRight(u)# é simétrica à
#flipLeft(u)#, quando os papéis de esquerda e direita estão invertidos. 
\codeimport{ods/RedBlackTree.flipRight(u)}

\subsection{Adição}

Para implementar
#add(x)# em uma #RedBlackTree#, fazemos uma inserção de um 
#BinarySearchTree# padrão para adicionar uma nova folha #u#, com $#u.x#=#x#$ e atribuímos 
$#u.colour#=#red#$.  Note que isso não muda a altura preta de nenhum nodo e, portanto, não viola a propriedade de altura preta.
Ela pode, entretanto, violar a propriedade de pender à esquerda (se #u# é o filho
à direita de seu pai) e pode violar a propriedade de nenhuma aresta vermelha (se o pai de #u# for #red#)
Para restaurar essas propriedades, chamamos o método #addFixup(u)#.
\codeimport{ods/RedBlackTree.add(x)}

Ilustrado na \figref{rb-addfix}, o método #addFixup(u)# recebe como entrada
um nodo #u# cuja cor é vermelha e que pode violar a propriedade de nenhuma
aresta vermelha e/ou a propriedade pendente à esquerda.
A discussão a seguir é provavelmente impossível de acompanhar sem 
observar a \figref{rb-addfix} ou recriá-la em um papel. 
Certamente, o leitor deve estudar essa figura antes de continuar a ler este capítulo.

\begin{figure}
  \begin{center}
    \includegraphics[width=\ScaleIfNeeded]{figs/rb-addfix}
  \end{center}
  \caption{Uma única rodada no processo de restaurar a Propriedade~2 após uma inserção.}
  \figlabel{rb-addfix}
\end{figure}

Se #u# for a raiz da árvore, então podemos pintar #u# de preto para reestabelecer as duas propriedades. Se 
o irmão de #u# também for vermelho, então o pai de #u# precisa ser preto, de forma que as propriedades de pender à esquerda e nenhuma aresta vermelha continuem valendo. 

Caso contrário, primeiro determinamos se o pai de #u#, que é #w#, viola a
propriedade de pender à esquerda e, se esse for o caso, realizamos uma operação
#flipLeft(w)# e atribuímos
$#u#=#w#$.  Isso nos deixa em um estado bem definido: #u# é o filho de seu pai
#w# então #w# agora satisfaz a propriedade de pender à esquerda.

Resta apenas assegurarmos que não há nenhuma aresta vermelha em #u#. Nós somente devemos nos preocupar com o caso em que #w# é vermelho, pois caso contrário #u#
satisfaz a propriedade de nenhuma aresta vermelha.

Como ainda não terminamos, #u# é vermelho e #w# é vermelho. A propriedade
de nenhuma aresta vermelha (que é violada por #u# e não por #w#) implica que 
o avô de #u#, o nodo #g#, existe e é preto. Se o filho à direita de #g# for
vermelho, então a propriedade de pender à esquerda garante que os dois filhos de #g# sejam vermelhos e uma chamada a 
#pushBlack(g)# torna #g# vermelho e #w# preto. Isso restaura
a propriedade nenhuma aresta vermelha em #u#, mas pode fazer que seja violada em #g# e, por isso, o processo reinicia com $#u#=#g#$.

Se o filho à direita de #g# for preto, então uma chamada a 
#flipRight(g)# faz 
#w# ser o pai (preto) de #g# e dá a #w# dois filhos vermelhos, #u# e 
#g#. Isso assegura que #u# satisfaz a propriedade de nenhuma aresta vermelha e #g#
satisfaz a propriedade de pender à esquerda. Nesse caso podemos parar.
\codeimport{ods/RedBlackTree.addFixup(u)}

O método
#insertFixup(u)# leva tempo constante por iteração e cada iteração ou termina ou move #u# mais próximo à raiz. 
Portanto, o método 
#insertFixup(u)# termina após $O(\log #n#)$ iterações em tempo 
$O(\log #n#)$.

\subsection{Remoção}

A operação 
#remove(x)# em uma #RedBlackTree# é a mais complicada para implementar 
e isso é verdade para todas as variantes conhecidas de árvores rubro-negras. 
Assim como a operação #remove(x)# em uma \texttt{BinarySearchTree},
essa operação se resume a achar um nodo #w# com somente um filho #u#
e remover #w# da árvore fazendo o #w.parent# adotar #u#.

O problema com isso é que, se #w# for preto, então a propriedade da altura preta 
será violada em #w.parent#. Podemos evitar esse problema, temporariamente, 
ao adicionar #w.colour# a #u.colour#. Isso causa dois outros problemas:
(1)~se #u# e #w# iniciaram-se pretos, então 
$#u.colour#+#w.colour#=2$ (preto duplo), que é uma cor inválida.
Se #w# era vermelho, então é substituído por um nodo preto #u#, o
que pode violar a propriedade de pender à esquerda em 
 $#u.parent#$.  Esse dois problemas podem ser resolvidos com uma
 chamada ao método #removeFixup(u)#.
\codeimport{ods/RedBlackTree.remove(x)}

O método #removeFixup(u)# recebe como entrada um nodo #u# cuja cor é preto
(1) ou preto duplo(2). Se #u# for preto duplo, então #removeFixup(u)#
realiza uma série de operações de rotações e alterações de cor que
movem o nodo preto duplo acima na árvore até que possa ser eliminado. Durante esse processo,
o nodo #u# muda até que, no final desse processo, #u#
refere-se à raiz da subárvore que foi alterada.
A raiz dessa subárvore pode ter mudado de cor. Em particular,
pode ter ido de vermelho para preto, então o método 
#removeFixup(u)# termina ao verificar se 
o pai de #u# viola a propriedade de pender à esquerda e, caso positivo, corrige esse problema. 
\codeimport{ods/RedBlackTree.removeFixup(u)}

O método
#removeFixup(u)# é ilustrado na \figref{rb-removefix}.
Novamente, o texto a seguir será difícil, senão impossível, de acompanhar sem 
observar a \figref{rb-removefix}.  Cada iteração do laço em 
#removeFixup(u)# processa o nodo preto duplo #u#, baseado em um de quatro casos: 

\begin{figure}
  \begin{center}
    \includegraphics[height=\HeightScaleIfNeeded]{figs/rb-removefix}
  \end{center}
  \caption{Uma rodada no processo de eliminar um nodo preto duplo após uma remoção.}
  \figlabel{rb-removefix}
\end{figure}

\noindent
Caso 0: #u# é a raiz. Esse é o caso mais fácil de tratar. Repintamos #u# em preto (isso não viola nenhuma das propriedades da árvore rubro-negra). 

\noindent 
Caso 1: o irmão de #u#, #v#, é vermelho. Nesse caso, o irmão de #u# é filho à esquerda de seu pai, #w# (de acordo com a propriedade de pender à esquerda).
Nós realizamos um 
right-flip em #w# e então procedemos à próxima iteração. Note que essa ação 
faz com que o pai de #w# viole a propriedade de pender à esquerda e faz a 
profundidade de #u# aumentar. Entretanto, isso também implica que a próxima iteração será no Caso~3 com #w# pintado de vermelho. Ao examinar o Caso~3 a seguir, veremos que o processo irá parar na próxima iteração. 
\codeimport{ods/RedBlackTree.removeFixupCase1(u)}

\noindent
Caso 2: o irmão de #u#, #v#, é preto, e #u# é o filho à esquerda de seu pai, #w#.
Nesse caso, chamamos 
#pullBlack(w)#, fazendo #u# preto,
#v# vermelho, e tornando #w# preto ou duplo preto. 
Nesse ponto, #w# não satisfaz a propriedade de pender à esquerda, então
chamamos #flipLeft(w)# para arrumar isso. 
Além disso, #w# é vermelho e #v# é a raiz da subárvore com que começamos.
Precisamos verificar se #w# faz a propriedade de não haver nenhuma aresta vermelha ser violada. Fazemos isso inspecionando o filho à direita de #w#, #q#.
Se #q# é preto, então #w# satisfaz a propriedade de nenhuma aresta vermelha e podemos continuar a próxima iteração com $#u#=#v#$.

Caso contrário (#q# é vermelho) então as duas propriedades de não haver nenhuma aresta vermelha e de pender à esquerda são violadas em #q# e #w#, respectivamente.

A propriedade de pender à esquerda é restaurada com uma chamada a 
#rotateLeft(w)#, mas a propriedade de não haver nenhuma aresta vermelha continua sendo violada. Nesse ponto, #q# é o filho à esquerda de #v#, #w# é o filho à esquerda de #q#, #q# e #w# são vermelhos e #v# é preto ou duplo preto. 
Uma #flipRight(v)# torna #q# o pai de #v# e de #w#.  
Em seguida, um #pushBlack(q)# faz #v#
e #w# pretos e devolve a cor de #q# de volta à cor original de #w#.

Nesse ponto, o nodo duplo preto foi eliminado e as propriedades  
nenhuma aresta vermelha e altura preta são restabelecidas.
Somente resta resolver um problema: o filho à direita de #v# pode ser vermelho e, nesse caso, a propriedade de pender à esquerda seria violada. Verificamos isso
e realizamos, se necessária uma correção, uma chamada a  
#flipLeft(v)#.
\codeimport{ods/RedBlackTree.removeFixupCase2(u)}

\noindent
Caso 3: o irmão de #u# é preto e #u# é o filho à direita de seu pai, #w#. 
Esse caso é simétrico ao Caso~2 e é tratado quase da mesma forma.
As únicas diferenças vêm do fato que a propriedade de 
pender à esquerda é assimétrica e, portanto, requer tratamento diferenciado.

Como antes, começamos com uma chamada a 
#pullBlack(w)#, que torna #v# vermelho e #u# preto. Uma chamada a 
#flipRight(w)# promove #v# à raiz da subárvore. Nesse momento, 
#w# é vermelho e existem duas formas de tratamento 
dependendo da cor do filho à esquerda de #w#, #q#.

Se #q# for vermelho, então o código termina exatamente da mesma maneira que o Caso~2,
mas é ainda mais simples pois não há perigo de #v# não 
satisfazer a propriedade de pender à esquerda.

O caso mais complicado ocorre quando #q# é preto. Nesse caso,
examinamos a cor do filho à esquerda de #v#. Se for vermelho, então #v# tem
dois filhos vermelhos e seu preto extra pode ser empurrado abaixo com uma 
chamada a 
#pushBlack(v)#.  Nesse ponto, #v# agora tem a cor original de #w# e terminamos o processo.
%%

Se o filho à esquerda de #v#'s for preto, então #v# viola a propriedade 
de pender à esquerda e a restauramos com uma chamada a 
#flipLeft(v)#.  Então retornamos o nodo #v# para que a próxima iteração de
#removeFixup(u)# então continue com 
$#u#=#v#$.
\codeimport{ods/RedBlackTree.removeFixupCase3(u)}.

Cada iteração de 
 #removeFixup(u)# leva tempo constante. Casos~2 e 3
 ou terminam ou movem #u# mais próximo à raiz da árvore. Caso~0 (o #u# é 
 a raiz) sempre termina e Caso~1 levam imediatamente ao Caso~3,
 que também termina. Como a
altura da árvore é no máximo $2\log
#n#$, concluímos o processo em até $O(\log #n#)$ iterações de 
#removeFixup(u)#, então #removeFixup(u)# roda em tempo $O(\log #n#)$.

\section{Resumo}
\seclabel{redblack-summary}

O teorema a seguir resume o desempenho de uma estrutura de dados 
#RedBlackTree#:

\begin{thm}
  Uma #RedBlackTree# implementa a interface #SSet# e possui as
  operações #add(x)#, #remove(x)# e #find(x)# em tempo $O(\log
  #n#)$ no pior caso, por operação.
\end{thm}

O bônus extra a seguir não está incluído no teorema anterior:

\begin{thm}\thmlabel{redblack-amortized}
Começando com uma #RedBlackTree# vazia, qualquer sequência de $m$ operações
  #add(x)# e #remove(x)# resulta em um total de tempo $O(m)$
  durante todas as chamadas #addFixup(u)# e #removeFixup(u)#. 
\end{thm}

Apenas rascunhamos uma prova do \thmref{redblack-amortized}. Ao comparar 
#addFixup(u)# e #removeFixup(u)# com os algoritmos para adicionar ou remover uma folha em uma 
árvore 2-4, podemos nos convencer que essa propriedade é herdada de uma 
 árvore 2-4.  Em particular, se podemos mostrar que o tempo total gasto com
 repartições de nodos, uniões e empréstimos em uma árvore 2-4 é $O(m)$ 
 então isso implica no \thmref{redblack-amortized}.

 A prova desse teorema para a árvore 2-4 usa o método potencial 
\index{método potencial}%
de análise amortizada.\footnote{Veja as prova do 
\lemref{dualarraydeque-amortized} e do \lemref{selist-amortized} para %%%%
outras aplicações do método potencial.} Defina o potencial de um nodo
interno #u# em uma árvore 2-4 como
\[
  \Phi(#u#) = 
    \begin{cases} 
      1 & \text{se #u# tem 2 filhos } \\ 
      0 & \text{se #u# tem 3 filhos} \\ 
      3 & \text{se #u# tem 4 filhos}  
    \end{cases}
\]
e o potencial de uma 
 árvore 2-4 como a soma dos potenciais de seus nodos. 
 Quando uma repartição ocorre, é porque um nodo com quatro filhos
 se torna dois nodos, com dois e três filhos. Isso significa que o potencial geral cai em 
$3-1-0 = 2$. Quando uma união ocorre, dois modos que tinham dois filhos
são substituídos por um nodo com três filhos. O resultado é uma queda em potencial
de $2-0=2$.  Portanto, para todas repartição ou união, o potencial diminui em dois.

A seguir note que se ignorarmos a repartição e união de nodos, há somente
um número constante de nodo cujo número de filhos é alterado pela adição
ou remoção de uma folha. Ao adicionar um nodo, algum nodo tem seu número
número de filhos aumentado em um, aumentando o potencial por até três.
Durante a remoção de uma folha, um nodo tem seu número de filhos reduzido
em um, aumentando o potencial em até um, e dois nodos podem estar envolvidos em
uma operação de empréstimo aumentando seu potencial total em até um.

Para resumir, cada união e repartição faz que o potencial caia por ao menos dois.
Ignorando uniões e repartições, cada adição ou remoção faz com que o 
potencial aumente em até três e o potencial sempre é não negativo.
Portanto, o número de repartições e uniões causados por $m$
adições ou remoções em uma árvore inicialmente vazia é até 
$3m/2$.
O \thmref{redblack-amortized} é uma consequência dessa análise e da 
correspondência entre árvores 2-4 e árvores rubro-negras.

\section{Discussão e Exercícios}

A árvore rubro-negra foi inicialmente desenvolvida por Guibas e Sedgewick \cite{gs78}.
Embora sua implementação seja de alta complexidade, ela é encontrada em algumas
das bibliotecas e aplicações mais utilizadas. A maior parte dos livros didáticos
de algoritmos e estruturas de dados discutem alguma variante de 
árvores rubro-negras.

Andersson \cite{a93} descreve uma versão pendente à esquerda de árvores 
balanceadas que é similar às árvores rubro-negras mas tem a 
restrição adicional de que todo nodo tem no máximo um filho vermelho. 
Isso implica que essas árvores simulam árvores 2-3 em vez de árvores 2-4. 
Ela são significantemente mais simples que a estrutura 
#RedBlackTree# apresentada neste capítulo.

Sedgewick \cite{s08} descreve duas versões de árvores rubro-negras 
pendentes à esquerda. Elas usam recursão juntamente com uma simulação 
\emph{top-down} de repartição e união em árvore 2-4. A combinação dessas 
duas técnicas tornam seus códigos especialmente curtos e elegantes.

Uma estrutura de dados relacionada e mais antiga é a \emph{árvore AVL}
\cite{avl62}.
\index{árvore AVL}%
Árvores AVL são \emph{balanceadas na altura}:
\index{balanceamento na altura}%
\index{árvore binária de busca!balanceamento na altura}%
A cada nodo $u$, a 
altura da subárvore enraizada em 
#u.left# e a subárvore enraizada em #u.right# difere em até um.
Segue imediatamente que se 
$F(h)$ é o número de folhas em uma árvore de altura $h$, então 
$F(h)$ segue a recorrência de Fibonacci 
\[
   F(h) = F(h-1) + F(h-2)
\]
com casos base $F(0)=1$ e $F(1)=1$.  Isso significa que $F(h)$ é aproximadamente  
$\varphi^h/\sqrt{5}$, onde $\varphi=(1+\sqrt{5})/2\approx1.61803399$ é a 
\emph{razão dourada}.  (Mais precisamente, $|\varphi^h/\sqrt{5} - F(h)|\le 1/2$.)
Seguindo de modo similar à prova do 
\lemref{twofour-height}, isso implica em 
\[
   h \le \log_\varphi #n# \approx 1.440420088\log #n# \enspace ,
\]
então as árvores AVL tem altura menor que árvore 
rubro-negras.  O balanceamento da altura pode ser mantido durante as operações
#add(x)# e #remove(x)# ao percorrer o caminho em direção à raiz realizando 
uma operação de rebalanceamento em cada nodo #u# onde a altura das subárvores
à esquerda e à direita de #u# difere em dois.
Veja a \figref{avl-rebalance}.

\begin{figure}
  \begin{center}
    \includegraphics[scale=0.90909]{figs/avl-rebalance}
  \end{center}
  \caption{Rebalanceamento em uma árvore AVL.  No máximo duas rotações são necessárias para converter um nodo cujas subárvores têm alturas de $h$ e $h+2$ em um nodo cujas subárvores têm alturas de até $h+1$}
  \figlabel{avl-rebalance}
\end{figure}

As variantes de árvores rubro-negras de Andersson e de Sedgewick
e as árvores AVL são mais simples de implementar que a estrutura #RedBlackTree#
aqui definida. Infelizmente nenhuma delas pode garantir que o tempo 
amortizado gasto rebalanceando é $O(1)$ por atualização. 
Em especial não há teorema análogo 
ao \thmref{redblack-amortized} para essas estruturas. 

\begin{figure}
  \centering{\includegraphics[scale=0.90909]{figs/redblack-example}}
  \caption{Uma árvore rubro-negra para praticar.}
  \figlabel{redblack-example2}
\end{figure}

\begin{exc}
  Desenhe a árvore 2-4 que corresponde à #RedBlackTree# na
  \figref{redblack-example2}.
\end{exc}

\begin{exc}
  Desenhe a adição de 13, então 3.5 e depois 3.3 na #RedBlackTree#
  na \figref{redblack-example2}.
\end{exc}

\begin{exc}
  Desenhe a remoção de 11, então 9 e depois 5 na #RedBlackTree# na
  \figref{redblack-example2}.
\end{exc}

\begin{exc}
  Mostre que, para valores arbitrariamente grandes de #n#, há árvores rubro-negras com #n# nodos com altura $2\log #n#-O(1)$.
\end{exc}

\begin{exc}
  Considere as operações #pushBlack(u)# e #pullBlack(u)#.  O que essas operações fazem com a 
árvore 2-4 implícita que está sendo simulada pela árvore rubro-negra? 
\end{exc}

\begin{exc}
Mostre que, para valores arbitrariamente grandes de #n#, existem sequências de operações 
#add(x)# e #remove(x)# que resultam em árvores rubro-negras com
  #n# nodos que têm altura $2\log #n#-O(1)$.
\end{exc}

\begin{exc}
Porque o método #remove(x)# na implementação da #RedBlackTree# realiza
a atribuição #u.parent=w.parent#?  Isso não seria feito pela chamada 
#splice(w)#?
\end{exc}

\begin{exc}
  Suponha que a árvore 2-4, $T$, tem $#n#_\ell$ folhas e $#n#_i$ nodos internos.
  \begin{enumerate}
    \item Qual é o valor mínimo de $#n#_i$, como uma função de $#n#_\ell$?
    \item Qual é o valor máximo de $#n#_i$, como uma função de $#n#_\ell$?
    \item Se $T'$ é uma árvore rubro-negra que representa $T$, então quantos nodos vermelhos $T'$ tem?
  \end{enumerate}
\end{exc}

\begin{exc}
Suponha que você tem uma árvore binária de busca com #n# nodos e uma altura de até 
$2\log #n#-2$. É sempre possível colorir os nodos vermelhos e pretos 
tal que a árvore satisfaz as propriedades da altura preta e de que não 
há nenhuma aresta vermelha? Se sim, também é possível fazê-la satisfazer 
a propriedade de pender à esquerda?
\end{exc}

\begin{exc}\exclabel{redblack-merge}
  Suponha que você tem duas árvores rubro-negras $T_1$ e $T_2$ que têm a
  mesma altura preta 
  , $h$, e tal que a maior chave em $T_1$ é menor que a menor chave em 
  $T_2$.  Mostre como juntar $T_1$ e $T_2$ em uma única árvore rubro-negra em
  tempo $O(h)$.
\end{exc}

\begin{exc}
  Estenda sua solução para \excref{redblack-merge} para o caso onde 
  as duas árvores
  $T_1$ e $T_2$ têm diferentes alturas pretas, $h_1\neq h_2$.
  O tempo de execução deve ser
   $O(\max\{h_1,h_2\})$.
\end{exc}

\begin{exc}
  Prove que, durante uma operação #add(x)#, uma árvore AVL deve realizar 
  até uma operação de rebalanceamento (que envolve até duas rotações;
  veja a \figref{avl-rebalance}).  Dê um exemplo de uma árvore AVL e uma operação 
  #remove(x)# nessa árvore que requer na ordem de $\log
  #n#$ operações de rebalanceamento. 
\end{exc}

\begin{exc}
  Implemente uma classe #AVLTree# que implementa árvores AVL conforme descrito acima. 
  Compare seu desempenho à implementação da #RedBlackTree#. 
  Qual implementação tem a operação #find(x)# mais rápida?
\end{exc}

\begin{exc}
  Projete e implemente uma série de experimentos que comparam o desempenho de 
  #find(x)#, #add(x)# e #remove(x)# para as implementações #SSet#: #SkiplistSSet#,
  #ScapegoatTree#, #Treap# e #RedBlackTree#.  Inclua diferentes cenários de teste,
  incluindo casos onde os dados são aleatórios, previamente ordenados, 
  são removidos em ordem aleatória, são removidos em ordem crescente 
  e assim por diante. 
\end{exc}
