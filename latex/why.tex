\chapter*{Por que este livro?}
\addcontentsline{toc}{chapter}{Por que este livro?}

Existem muitos livros introdutórios à disciplina de estruturas de dados.
Alguns deles são muito bons. A maior parte deles são pagos, e a
grande maioridade de estudantes de Ciência da Computação e Sistemas de
Informação irá gastar algum dinheiro em um livro de estruturas de dados.

Vários livros gratuitos sobre estruturas de dados estão disponíveis online.
Alguns muito bons, mas a maior parte está ficando desatualizada. Grande
parte deles torna-se gratuita quando seus autores e/ou editores decidem
parar de atualizá-los. Atualizar esses livros frequentemente não é possível,
por duas razões: (1)~Os direitos autorais pertencem ao autor e/ou editor,
os quais podem não autorizar tais atualizações. (2)~O \emph{código-fonte} desses
livros muitas vezes não está disponível. Isto é, os arquivos Word, WordPerfect,
FrameMaker ou \LaTeX\ para o livro não são acessíveis e, ainda, a versão do
software que processa tais arquivos pode não estar mais disponível.

A meta deste projeto é libertar estudantes de graduação de Ciência da Computação
de ter que pagar por um livro introdutório à disciplina de estruturas de dados.

Eu
\footnote{The translator to Portuguese language is deeply grateful to the original author of this book Pat Morin for his decision which allows the availability of a good quality and free book in the
native language of my Brazilian students.} decidi implementar essa meta ao tratar esse livro como um projeto de Software Aberto
\index{Open Source}%
\index{Software Aberta}%
.

Os arquivos-fonte originais em \LaTeX, \lang\ e scripts para montar este livro estão disponíveis para download a partir do website do autor\footnote{\url{http://opendatastructures.org}}
e também, de modo mais importante, em um site confiável de gerenciamento de códigos-fonte
.\footnote{\url{https://github.com/patmorin/ods}}\footnote{Tradução em português em \url{https://github.com/albertiniufu/ods}}

O código-fonte disponível é publicado sob uma licença
Creative Commons Attribution,
o que quer dizer que qualquer um é livre para \emph{compartilhar}:
\index{compartilhar}
copiar, distribuir e
transmitir essa obra; e \emph{modificar}:
\index{modificar}
adaptar a obra, incluindo o direito
de fazer uso comercial da obra.
A única condição a esse direitos é \emph{atribuição}:
você deve reconhecer que a obra derivada contém código e/ou texto de \url{opendatastructures.org}.

Qualquer pessoa pode contribuir com correções usando o sistema de gerenciamento
de códigos-fonte \texttt{git}
\index{git@\texttt{git}}
. Qualquer pessoa também pode fazer fork (criar versão alternativa) dos arquivos-fonte deste livro e desenvolver uma versão separada (por exemplo, em outra linguagem de programação).
Minha esperança é que, ao assim fazer, este livro continuará a
ser didático mesmo depois que o meu interesse no projeto, ou meu batimento cardíaco, desapareça (o que quer que aconteça primeiro).

